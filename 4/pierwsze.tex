\subsection{Węzły pierwsze}

Okazuje się, że liczby pierwsze znane z teorii liczb mają swój odpowiednik wśród węzłów.
Odpowiednik ten jest ściśle związany z operacją sumy spójnej, którą teraz zdefiniujemy.

\begin{definicja}
Kiedy $A$ i $B$, to ich sumą spójną $A \# B$ jest węzeł, który powstaje z poprzednich przez zszycie (?).
\end{definicja}

Okazuje się, że tak określone działanie jest dobrze określone, to znaczy nie zależy od wyboru łuków zszywających.
Możemy przejść do definicji węzłów pierwszych.

\begin{definicja}
Węzeł nazywamy pierwszym, kiedy nie jest niewęzłem, a jednocześnie nie daje się zapisać jako suma spójna dwóch węzłów, które nie są niewęzłami.
Węzeł, który nie jest pierwszy, nazywamy złożonym (niewęzeł?).
\end{definicja}

Ze względu na niedostatecznie rozwinięty aparat matematyczny nie możemy podać dowodu następującego faktu.

\begin{twierdzenie}[Schubert, 1949]
Każdy węzeł rozkłada się jednoznacznie na węzły pierwsze (z dokładnością do kolejności składników).
\end{twierdzenie}

Istnieje nieskończenie wiele węzłów pierwszych: takie są wszystkie węzły torusowe.