% Remigiusz
\section{Wielomian Jonesa}
W tej sekcji zbadamy inny wielomianowy niezmiennik węzłów, wielomian Jonesa.
Został odkryty w 1984 roku przez Vaughana Jonesa, znajduje zastosowanie między innymi przy badaniu węzłów przemiennych.

\subsection{Nawias Kauffmana}
Zaczniemy od zdefiniowania nawiasu Kauffmana.
Przypomnijmy, wielomian Laurenta zmiennej $X$ to formalny symbol $f=a_r X^r + \ldots + a_s X^s$, gdzie $r, s, a_r, \ldots, a_s$ są całkowite i $r \le s$.

\begin{definicja}
	\emph{Nawias Kauffmana} $\langle D \rangle$ dla diagramu splotu $D$ to wielomian Laurenta zmiennej $A$, który jest niezmienniczy ze względu na gładkie deformacje diagramu, a przy tym spełnia trzy aksjomaty:
	\begin{enumerate}
		\item $\langle \NieWezel \rangle=1$
		\item $\langle D \sqcup \NieWezel \rangle = (-A^{-2} - A^2) \langle D \rangle$
		\item $\langle \PrawyKrzyz \rangle = A \langle \PrawyGladki \rangle + A^{-1} \langle \LewyGladki  \rangle$
	\end{enumerate}
\end{definicja}

Tutaj $\NieWezel$ oznacza standardowy diagram dla niewęzła, $D \sqcup \NieWezel$ jest diagramem, który powstaje z $D$ przez dodanie nieprzecinającej go krzywej zamkniętej, zaś trzy symbole $\PrawyKrzyz$, $\PrawyGladki$ oraz $\LewyGladki $ odnoszą się do diagramów, które są identyczne wszędzie poza małym obszarem.
Diagramy $\PrawyGladki$ oraz $\LewyGladki$ nazywa się odpowiednio dodatnim (prawym) i ujemnym (lewym) wygładzeniem $\PrawyKrzyz$

\begin{lemat}
	Nawias Kauffmana dowolnego diagramu można wyznaczyć w skończonym czasie.
\end{lemat}

\begin{proof}
	Jeżeli diagram $D$ ma $n$ skrzyżowań, to nieustanne stosowanie aksjomatu trzeciego pozwala na zapisanie $\langle D \rangle$ jako sumy $2^n$ składników, z których każdy jest po prostu zamkniętą krzywą i ma trywialny nawias ($\langle \MalyNieWezel \rangle = 1$).
	Nawias sumy wyznacza się korzystając z drugiego aksjomatu.
\end{proof}

Przedstawimy teraz wpływ ruchów Reidemeistera na nawias Kauffmana.

\begin{lemat}
	Pierwszy ruch Reidemeistera zmienia nawias Kauffmana zgodnie z poniższą regułą.
	Pozosałe ruchy Reidemeistera nie zmieniają nawiasu.
	\[
		% pierwszy ruch Reidemeistera
		\left\langle\begin{tikzpicture}[scale=0.05, baseline=-6]
			\clip (-12,-12) rectangle (1,7);
			\path[ARC] (-10,7) .. controls (-10,3) and (-10,0) .. (-6,-4);
			\path[ARC] (-6,0) .. controls (2,8) and (2,-10) .. (-6,-4);
			\path[ARC] (-10,-11) .. controls (-10,-8) and (-10,-5) .. (-9,-4);
		\end{tikzpicture}\right\rangle
		= -A^{-3}
		\left\langle\ \begin{tikzpicture}[scale=0.05,baseline=-6]
			\path[ARC] (-10,7) -- (-10,-11);
		\end{tikzpicture}\ \right\rangle
		\,\bullet\,
		% drugi ruch Reidemeistera
		\left\langle\begin{tikzpicture} [scale=0.04,baseline=-5]
			\path[ARC](0,10) .. controls (10,5) and (10,-9) .. (0,-14);
			\path[ARC] (10,10) .. controls (8,9) .. (7,8);
			\path[ARC] (3,4.5) .. controls (-1,0) and (-1,-4) .. (3,-8);
			\path[ARC] (10,-14) .. controls (8,-13) .. (7,-12);
		\end{tikzpicture}\right\rangle
		=
		\left\langle\ \begin{tikzpicture} [scale=0.04, baseline=-5]
		\path[ARC] (0,10) .. controls (3,8) and (3,0) .. (3,-2) .. controls (3,-4) and (3,-12) .. (0,-14);
		\path[ARC] (10,10) .. controls (7,8) and (7,0) .. (7,-2) .. controls (7,-4) and (7,-12) .. (10,-14);
		\end{tikzpicture}\ \right\rangle
		\,\bullet\,
		% trzeci ruch Reidemeistera
		\left\langle\begin{tikzpicture} [scale=0.04, auto, baseline=-6] %Reidemeister 3 left
			\path[ARC] (-10,10) -- (-6.6,6);
			\path[ARC] (-4,3) -- (10,-14);
			\path[ARC] (10,10) -- (6.6,6);
			\path[ARC] (4,3) -- (1.6,0);
			\path[ARC] (-1.6,-4) -- (-10,-14);
			\path[ARC] (-14,-2) .. controls (-6, -2) and (-6,8) .. (0,8);
			\path[ARC] (14,-2) .. controls (6, -2) and (6,8) .. (0,8);
		\end{tikzpicture}\right\rangle
		=
		\left\langle\begin{tikzpicture} [scale=0.04, auto, baseline=-6] %Reidemeister 3 right
			\begin{scope}[xshift=1300,rotate=180,yshift=110]
				\path[ARC] (-10,10) -- (-6.6,6);
				\path[ARC] (-4,3) -- (10,-14);
				\path[ARC] (10,10) -- (6.6,6);
				\path[ARC] (4,3) -- (1.6,0);
				\path[ARC] (-1.6,-4) -- (-10,-14);
				\path[ARC] (-14,-2) .. controls (-6, -2) and (-6,8) .. (0,8);
				\path[ARC] (14,-2) .. controls (6, -2) and (6,8) .. (0,8);
			\end{scope}
		\end{tikzpicture}\right\rangle.
	\]
\end{lemat}

\begin{proof}
Pierwszy ruch Reidemeistera:
\[
\left\langle\begin{tikzpicture}[scale=0.05, baseline=-6] %Reidemeister 1
	\clip (-12,-12) rectangle (1,7);
	\path[ARC] (-10,7) .. controls (-10,3) and (-10,0) .. (-6,-4);
	\path[ARC] (-6,0) .. controls (2,8) and (2,-10) .. (-6,-4);
	\path[ARC] (-10,-11) .. controls (-10,-8) and (-10,-5) .. (-9,-4);
\end{tikzpicture}\right\rangle
\stackrel{K3}{=}
A\left\langle\begin{tikzpicture}[scale=0.05, baseline=-6] %Reidemeister 1
	\clip (-12,-12) rectangle (1,7);
	\path[ARC] (-10,7) .. controls (-10,3) and (-9,-3) .. (-6,0);
	\path[ARC] (-6,0) .. controls (2,8) and (2,-10) .. (-6,-4);
	\path[ARC] (-10,-11) .. controls (-10,-8) and (-10,-1) .. (-6,-4);
\end{tikzpicture}\right\rangle
+A^{-1}\left\langle
\begin{tikzpicture}[scale=0.05, baseline=-6] %Reidemeister 1
	\clip (-12,-12) rectangle (1,7);
	\path[ARC] (-10,7) .. controls (-10,3) and (-7,-2) .. (-9,-4);
	\path[ARC] (-5,1) .. controls (2,8) and (2,-10) .. (-5,-5);
	\path[ARC] (-5,1) .. controls (-6.5,-0.5) and (-6.5,-3.5) .. (-5,-5);
	\path[ARC] (-10,-11) .. controls (-10,-8) and (-10,-5) .. (-9,-4);
\end{tikzpicture}\right\rangle
\stackrel{K2}{=}
A\left\langle\ 
\begin{tikzpicture}[scale=0.05, baseline=-6] %Reidemeister 1
	\path[ARC] (-10,7) -- (-10,-11);
\end{tikzpicture}\ 
\right\rangle
+A^{-1}(-A^{-2}-A^2)\left\langle\ 
\begin{tikzpicture}[scale=0.05, baseline=-6] %Reidemeister 1
	\path[ARC] (-10,7) -- (-10,-11);
\end{tikzpicture}
\ \right\rangle
=
-A^{-3}\left\langle\ 
\begin{tikzpicture}[scale=0.05, baseline=-6] %Reidemeister 1
	\path[ARC] (-10,7) -- (-10,-11);
\end{tikzpicture}\ 
\right\rangle
\]
Pierwsza równość wynika z $K3$, druga z $K2$, trzecia jest oczywista.
Dla drugiego ruchu:
\begin{eqnarray*}
\left\langle\begin{tikzpicture} [scale=0.04,baseline=-5] %Reidemeister 2
	\path[ARC](0,10) .. controls (10,5) and (10,-9) .. (0,-14);
	\path[ARC] (10,10) .. controls (8,9) .. (7,8);
	\path[ARC] (3,4.5) .. controls (-1,0) and (-1,-4) .. (3,-8);
	\path[ARC] (10,-14) .. controls (8,-13) .. (7,-12);
\end{tikzpicture}\right\rangle
&\stackrel{K3}{=}&
\phantom{-}A^{\phantom{-2}}
\left\langle\begin{tikzpicture} [scale=0.04,baseline=-5] %Reidemeister 2
	\clip (-2,-14) rectangle (12,10);
	\path[ARC] (0,10) .. controls (4.5,6) and (5.5,6) .. (10,10);
	\path[ARC] (0,-14) .. controls (4,-11) .. (6,-8) .. controls (12,0) and (-5,0) .. (3,-8);
	\path[ARC] (10,-14) .. controls (8.5,-13) .. (7,-12);
\end{tikzpicture}\right\rangle
+
\phantom{A}A^{-1}
\left\langle\begin{tikzpicture} [scale=0.04,baseline=-5] %Reidemeister 2
	\clip (-2,-14) rectangle (12,10);
	\path[ARC] (10,10) .. controls (8,9) .. (7,8);
	\path[ARC] (10,-14) .. controls (8,-13) .. (7,-12);
	\path[ARC](10,10) .. controls (0,5) and (15,-4) .. (0,-14);	
	\path[ARC] (0,10) .. controls (10,5) and (-3,0) .. (3,-8);
\end{tikzpicture}\right\rangle
\stackrel{K1}{=}
-A^{-2}
\left\langle\begin{tikzpicture} [scale=0.04, baseline=-5]
	\clip (-2,-14) rectangle (12,10);
	\path[ARC] (0,10) .. controls (4.5,6) and (5.5,6) .. (10,10);
	\path[ARC] (0,-14) .. controls (4.5,-10) and (5.5,-10) .. (10,-14);
\end{tikzpicture}\right\rangle
+
\phantom{A}
A^{-1}
\left\langle\begin{tikzpicture} [scale=0.04,baseline=-5] %Reidemeister 2
	\clip (-2,-14) rectangle (12,10);
	\path[ARC] (10,10) .. controls (8,9) .. (7,8);
	\path[ARC] (10,-14) .. controls (8,-13) .. (7,-12);
	\path[ARC](10,10) .. controls (0,5) and (15,-4) .. (0,-14);	
	\path[ARC] (0,10) .. controls (10,5) and (-3,0) .. (3,-8);
\end{tikzpicture}\right\rangle
\\
&\stackrel{K3}{=}&
-A^{-2}
\left\langle\begin{tikzpicture} [scale=0.04, baseline=-5]
	\clip (-2,-14) rectangle (12,10);
	\path[ARC] (0,10) .. controls (4.5,6) and (5.5,6) .. (10,10);
	\path[ARC] (0,-14) .. controls (4.5,-10) and (5.5,-10) .. (10,-14);
\end{tikzpicture}\right\rangle
+A^{-1}A
\left\langle\begin{tikzpicture} [scale=0.04, baseline=-5] %Reidemeister 2
	\clip (-2,-14) rectangle (12,10);
	\path[ARC] (0,10) .. controls (3,8) and (3,0) .. (3,-2) .. controls (3,-4) and (3,-12) .. (0,-14);
	\path[ARC] (10,10) .. controls (7,8) and (7,0) .. (7,-2) .. controls (7,-4) and (7,-12) .. (10,-14);
\end{tikzpicture}\right\rangle
+
A^{-1}A^{-1}
\left\langle\begin{tikzpicture} [scale=0.04, baseline=-5]
	\clip (-2,-14) rectangle (12,10);
	\path[ARC] (0,10) .. controls (4.5,6) and (5.5,6) .. (10,10);
	\path[ARC] (0,-14) .. controls (4.5,-10) and (5.5,-10) .. (10,-14);
\end{tikzpicture}\right\rangle
=%\phantom{-A^{-2}}
\left\langle\begin{tikzpicture} [scale=0.04, baseline=-5] %Reidemeister 2
	\clip (-2,-14) rectangle (12,10);
	\path[ARC] (0,10) .. controls (3,8) and (3,0) .. (3,-2) .. controls (3,-4) and (3,-12) .. (0,-14);
	\path[ARC] (10,10) .. controls (7,8) and (7,0) .. (7,-2) .. controls (7,-4) and (7,-12) .. (10,-14);
\end{tikzpicture}\right\rangle
\end{eqnarray*}
Dla trzeciego ruchu:
\begin{eqnarray*}
\left\langle\begin{tikzpicture} [scale=0.04, auto, baseline=-6] %Reidemeister 3 left
	\path[ARC] (-10,10) -- (-6.6,6);
	\path[ARC] (-4,3) -- (10,-14);
	\path[ARC] (10,10) -- (6.6,6);
	\path[ARC] (4,3) -- (1.6,0);
	\path[ARC] (-1.6,-4) -- (-10,-14);
	\path[ARC] (-14,-2) .. controls (-6, -2) and (-6,8) .. (0,8);
	\path[ARC] (14,-2) .. controls (6, -2) and (6,8) .. (0,8);
\end{tikzpicture}\right\rangle
&\stackrel{K3}{=}&
A
\left\langle\begin{tikzpicture} [scale=0.04, auto, baseline=-6] %Reidemeister 3 left
	\path[ARC] (-10,10) -- (-6.6,6);
	\path[ARC] (10,10) -- (6.6,6);
	\path[ARC] (-14,-2) .. controls (-6, -2) and (-6,8) .. (0,8);
	\path[ARC] (14,-2) .. controls (6, -2) and (6,8) .. (0,8);
	\path[ARC] (-10,-14) .. controls (0,-3) and (0,-3) .. (10,-14);
	\path[ARC] (-4,3) .. controls (0,-2) .. (4,3);
\end{tikzpicture}\right\rangle
+A^{-1}
\left\langle\begin{tikzpicture} [scale=0.04, auto, baseline=-6] %Reidemeister 3 left
	\path[ARC] (-14,-2) -- (14,-2);
	\path[ARC] (-10,10) .. controls (-7,6) and (-4,4) .. (-4,0);
	\path[ARC] (-10,-14) .. controls (-7,-10) and (-4,-8) .. (-4,-4);
	\path[ARC] (10,10) .. controls (7,6) and (4,4) .. (4,0);
	\path[ARC] (10,-14) .. controls (7,-10) and (4,-8) .. (4,-4);
\end{tikzpicture}\right\rangle
\stackrel{R2}{=}
A
\left\langle\begin{tikzpicture} [scale=0.04, auto, baseline=-6] %Reidemeister 3 left
	\path[ARC] (-14,-2) -- (14,-2);
	\path[ARC] (-10,-14) .. controls (0,-3) and (0,-3) .. (10,-14);
	\path[ARC] (-10,10) .. controls (0,-1) and (0,-1) .. (10,10);
\end{tikzpicture}\right\rangle
+A^{-1}
\left\langle\begin{tikzpicture} [scale=0.04, auto, baseline=-6] %Reidemeister 3 left
	\path[ARC] (-14,-2) -- (14,-2);
	\path[ARC] (-10,10) .. controls (-7,6) and (-4,4) .. (-4,0);
	\path[ARC] (-10,-14) .. controls (-7,-10) and (-4,-8) .. (-4,-4);
	\path[ARC] (10,10) .. controls (7,6) and (4,4) .. (4,0);
	\path[ARC] (10,-14) .. controls (7,-10) and (4,-8) .. (4,-4);
\end{tikzpicture}\right\rangle
\\
&\stackrel{R2}{=}&
A
\left\langle\begin{tikzpicture} [scale=0.04, auto, baseline=-6,yscale=-1] %Reidemeister 3 left
	\path[ARC] (-10,10) -- (-6.6,6);
	\path[ARC] (10,10) -- (6.6,6);
	\path[ARC] (-14,-2) .. controls (-6, -2) and (-6,8) .. (0,8);
	\path[ARC] (14,-2) .. controls (6, -2) and (6,8) .. (0,8);
	\path[ARC] (-10,-14) .. controls (0,-3) and (0,-3) .. (10,-14);
	\path[ARC] (-4,3) .. controls (0,-2) .. (4,3);
\end{tikzpicture}\right\rangle
+A^{-1}
\left\langle\begin{tikzpicture} [scale=0.04, auto, baseline=-6] %Reidemeister 3 left
	\path[ARC] (-14,-2) -- (14,-2);
	\path[ARC] (-10,10) .. controls (-7,6) and (-4,4) .. (-4,0);
	\path[ARC] (-10,-14) .. controls (-7,-10) and (-4,-8) .. (-4,-4);
	\path[ARC] (10,10) .. controls (7,6) and (4,4) .. (4,0);
	\path[ARC] (10,-14) .. controls (7,-10) and (4,-8) .. (4,-4);
\end{tikzpicture}\right\rangle
\stackrel{K3}{=}
\left\langle\begin{tikzpicture} [scale=0.04, auto, baseline=-6] %Reidemeister 3 right
\begin{scope}[xshift=1300,rotate=180,yshift=110]
	\path[ARC] (-10,10) -- (-6.6,6);
	\path[ARC] (-4,3) -- (10,-14);
	\path[ARC] (10,10) -- (6.6,6);
	\path[ARC] (4,3) -- (1.6,0);
	\path[ARC] (-1.6,-4) -- (-10,-14);
	\path[ARC] (-14,-2) .. controls (-6, -2) and (-6,8) .. (0,8);
	\path[ARC] (14,-2) .. controls (6, -2) and (6,8) .. (0,8);
\end{scope}
\end{tikzpicture}\right\rangle
\end{eqnarray*}
korzystaliśmy tu z własności drugiego ruchu.
\end{proof}

\subsection{Spin}
Kolejnym składnikiem jest spin.

\begin{definicja}
	Niech $D$ będzie diagramem zorientowanego splotu lub węzła.
	\textbf{Spinem} $D$ jest $w(D) = \sum_c \operatorname{sign} c$, gdzie sumowanie przebiega po wszystkich skrzyżowaniach.
\end{definicja}

\[
\begin{tikzpicture}[scale=0.075]
	\clip (-16,-16) rectangle (16,17); %No labels
	\foreach \x in {270,30, 150}
		\path[ARC,->-] (15+\x:6) .. controls (130+\x:25) and (200+\x:25) .. (225+\x:10);
	\node[red] (C1) at (30:14) {\small $+1$};
	\node[red] (C2) at (150:14) {\small $+1$};
	\node[red] (C3) at (270:14) {\small $+1$};
	\node[darkblue] at (0,0) {$D_1$};
\end{tikzpicture}
\]

\begin{lemat}
Pierwszy ruch Reidemeistera zmniejsza spin o jeden, pozostałe nie mają wpływu.
\[
w\left(\begin{tikzpicture}[scale=0.05, baseline=-6] %Reidemeister 1
	\clip (-12,-12) rectangle (1,7);
	\path[ARC] (-10,7) .. controls (-10,3) and (-10,0) .. (-6,-4);
	\path[ARC] (-6,0) .. controls (2,8) and (2,-10) .. (-6,-4);
	\path[ARC] (-10,-11) .. controls (-10,-8) and (-10,-5) .. (-9,-4);
\end{tikzpicture}\right)
=
w\left(\ \begin{tikzpicture}[scale=0.05,baseline=-6] %Reidemeister 1
	\path[ARC] (-10,7) -- (-10,-11);
\end{tikzpicture}\ \right)-1
\]
\end{lemat}

\begin{proof}
Proste ćwiczenie.
\end{proof}

\subsection{Wielomian Jonesa}

\begin{definicja}
\emph{Wielomian Jonesa} zorientowanego splotu to wielomian Laurenta $V(L)\in\Z[t^{1/2},t^{-1/2}]$ określony przez
\[V(L)=\left[ (-A)^{-3w(D)}\langle D\rangle\right]_{t^{1/2}=A^{-2}},\]
gdzie $D$ to dowolny diagram dla $L$.
\end{definicja}

\begin{twierdzenie}
Wielomian Jonesa jest niezmiennikiem zorientowanych splotów.
\end{twierdzenie}

\begin{proof}
Skorzystamy z tego, że indeks zaczepienia jest niezmiennikiem.
Wystarczy pokazać niezmienniczość $(-A)^{-3w(D)}\langle D\rangle$ na ruchy Reidemeistera.
ale
\begin{eqnarray*}
(-A)^{-3
w\left(\begin{tikzpicture}[scale=0.03, baseline=-4] %Reidemeister 1
	\clip (-12,-12) rectangle (1,7);
	\path[ARC] (-10,7) .. controls (-10,3) and (-10,0) .. (-6,-4);
	\path[ARC] (-6,0) .. controls (2,8) and (2,-10) .. (-6,-4);
	\path[ARC] (-10,-11) .. controls (-10,-8) and (-10,-5) .. (-9,-4);
\end{tikzpicture}\right)}
\left\langle\begin{tikzpicture}[scale=0.05, baseline=-6] %Reidemeister 1
	\clip (-12,-12) rectangle (1,7);
	\path[ARC] (-10,7) .. controls (-10,3) and (-10,0) .. (-6,-4);
	\path[ARC] (-6,0) .. controls (2,8) and (2,-10) .. (-6,-4);
	\path[ARC] (-10,-11) .. controls (-10,-8) and (-10,-5) .. (-9,-4);
\end{tikzpicture}\right\rangle
&=&
(-A)^{-3
w\left(\ \begin{tikzpicture}[scale=0.03,baseline=-4] %Reidemeister 1
	\path[ARC] (-10,7) -- (-10,-11);
\end{tikzpicture}\ \right)+3}
(-A)^{-3}
\left\langle\ \begin{tikzpicture}[scale=0.05,baseline=-6] %Reidemeister 1
	\path[ARC] (-10,7) -- (-10,-11);
\end{tikzpicture}\ \right\rangle\\
&=&
(-A)^{-3
w\left(\ \begin{tikzpicture}[scale=0.03,baseline=-4] %Reidemeister 1
	\path[ARC] (-10,7) -- (-10,-11);
\end{tikzpicture}\ \right)}
\left\langle\ \begin{tikzpicture}[scale=0.05,baseline=-6] %Reidemeister 1
	\path[ARC] (-10,7) -- (-10,-11);
\end{tikzpicture}\ \right\rangle.\qedhere
\end{eqnarray*}
\end{proof}

\subsection{Relacja kłębiasta}
Dotychczas wyznaczyliśmy wielomian Jonesa jedynie dla trywialnych splotów.
Spowodowane jest to tym, że chociaż nawias Kauffmana jest przydatny przy dowodzeniu różnych własności, to zupełnie nie nadaje się do obliczeń.
Dużo lepszym narzędziem jest następujące twierdzenie.

\begin{twierdzenie}[relacja kłębiasta]
	Wielomian Jonesa spełnia równość $V(\NieWezel) = 1$ oraz relację
	\[
		t^{-1} V(L_+) - tV(L_-) + (t^{-1/2} - t^{1/2}) V(L_0) = 0,
	\]

	gdzie $L_+$, $L_-$, $L_0$ to zorientowane sploty, kóre różnią się jedynie na małym obszarze:
	$
		\begin{tikzpicture}[scale=0.03, baseline=-3]
		\path[TEXTARC,->] (-5,-5) -- (5,5);
		\path[TEXTARC] (5,-5) -- (1.5,-1.5);
		\path[TEXTARC,<-] (-5,5) -- (-1.5,1.5);
		\end{tikzpicture}
	$, 
	$
		\begin{tikzpicture}[scale=0.03, baseline=-3]
		\path[TEXTARC,->] (5,-5) -- (-5,5);
		\path[TEXTARC] (-5,-5) -- (-1.5,-1.5);
		\path[TEXTARC,<-] (5,5) -- (1.5,1.5);
		\end{tikzpicture}
	$, 
	$
		\begin{tikzpicture}[scale=0.03, baseline=-3]
		\path[TEXTARC,->] (-5,-5) .. controls (-1.5,-1.5) and (-1.5,1.5) .. (-5,5);
		\path[TEXTARC,->] (5,-5) .. controls (1.5,-1.5) and (1.5,1.5) .. (5,5);
		\end{tikzpicture}
	$.
\end{twierdzenie}

\begin{proof}
Wyraźmy wielomian Jonesa przez nawias Kauffmana i spin.
Chcemy pokazać, że
\[
A^{4}(-A)^{-3w(L_+)}\langle\PrawyKrzyz\rangle
-A^{-4}(-A)^{-3w(L_-)}\langle\LewyKrzyz\rangle
+(A^2-A^{-2})(-A)^{-3w(L_0)}\langle\PrawyGladki\rangle
=0.
\]
Ale $w(L_\pm)=w(L_0)\pm 1$, zatem to jest równoważne z $-A\langle\PrawyKrzyz\rangle +A^{-1}\langle\LewyKrzyz\rangle +(A^2-A^{-2})\langle\PrawyGladki\rangle =0$.
Z definicji nawiasu Kauffmana wnioskujemy, że
$\langle\PrawyKrzyz\rangle = A\langle\PrawyGladki\rangle+A^{-1}\langle\LewyGladki\rangle$ i $\langle\LewyKrzyz\rangle = A\langle\LewyGladki\rangle+A^{-1}\langle\PrawyGladki\rangle$.
Pierwsze równanie przemnóżmy przez $A$, drugie przez $A^{-1}$, a następnie dodajmy je do siebie.
Wtedy otrzymamy $A\langle\PrawyKrzyz\rangle-A^{-1}\langle\LewyKrzyz\rangle = A^2\langle\PrawyGladki\rangle - A^{-2}\langle\PrawyGladki\rangle$.
\end{proof}

Przykład.
	$V(
		\begin{tikzpicture}
		[scale=0.02, baseline=-3]
		\clip (-15,-9) rectangle (15,9);
		\path[TEXTARC,-<-] (1.5,-2.75) arc (-20:300:8);
		\path[TEXTARC,-<-] (-1.5,2.75) arc (160:480:8);
		\end{tikzpicture}) = -t^{5/2} - t ^{1/2}
	$ (splot Hopfa),
	$
	V(\begin{tikzpicture}[scale=0.02,baseline=-2]
	\clip (-19,-13) rectangle (19,17);
	\foreach \x in {270,30, 150}
		\path[TEXTARC,-<-] (15+\x:6) .. controls (130+\x:25) and (200+\x:25) .. (225+\x:10);
\end{tikzpicture}) = -t^4+t^3 + t
$ (trójlistnik).

\subsection{Odwrotności, lustra i sumy}

\begin{twierdzenie}
Niech $L$ będzie zorientowanym splotem.
$V(rL)=V(L)$, $V(mL)(t)=V(L)(t^{-1})$.
\end{twierdzenie}

\begin{wniosek}
Wielomian Jonesa nie zależy od orientacji węzła (ale nie splotu!).
\end{wniosek}

\begin{wniosek}
Trójlistnik nie jest równoważny ze swoim lustrem.
\end{wniosek}

\begin{twierdzenie}
Niech $L, M$ będą zorientowanymi splotami, zaś $J, K$: zorientowanymi węzłami.
\begin{enumerate}
\item $V(L \sqcup M) = (-t^{1/2} - t^{-1/2}) V(L) V(M)$,
\item $V(J \# K) = V(J) V(K)$.
\end{enumerate}
\end{twierdzenie}