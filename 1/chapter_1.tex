% Maciek
\section{Wstęp, czyli trochę zakłamanej historii i intuicji}
Pojęcie węzła wywodzi się z teorii atomu Lorda Kelvina, która głosi, że atomy powstają z wirów energii (cokolwiek miałoby to znaczyć). Takie ``wiry`` Kelvin 
wyobrażał sobie jako okręgi w przestrzeni trójwymiarowej, lub, bardziej ogólnie, jako krzywe zamknięte w $\mathbb{R}^3$ bez samoprzecięc. Po tym, jak owa teoria się przyjęła 
(co prawda nie na długo), zaczęto szukać języka, w którym możnaby w miarę precyzyjnie (jak na tamte matematycznie prehistoryczne czasy) ją wyrazić. Jak mniemam (autor świadomie
używa nieco wycofanej formy ''jak mniemam'', zamiast np. ''z pewnością'', co pozwala mu nie wdawać się w dyskusję na temat historii matematyki, o której nie ma zielonego pojęcia)
te rozważania dały początek teorii węzłów. 

Praca Petera Taita, problemy, które naptokał, okazało się, że wiele z nich jest równoważnych temu, czy węzeł nie jest równoważny niewęzłowi, próba sklasyfikowania wszystkich 
węzłów o niewielkiej liczbie przecięć.

Trochę historii (najpierw Alexander, potem Reidemeister, potem jeszcze ktoś... dobra, to miał być szkielet)

Intuicja, kilka wstępnych przykładów... (muszę się nauczyć tej magicznej techologii służącej do rysowania tych... węzłów)

\section{Rozdział Pierwszy}
 
Jest dość dużo (więcej niż trzy) definicji węzła. Każda z nich w pewnym sensie oddaję intuicję, która kryje się za potocznym rozumieniem nazwy tego pojęcia. W tym rozdziale podamy
jedną z nich.
Powiemy też, co to znaczy, że dwa węzły są równoważne, tj. zastanowimy się, kiedy mając dane dwa węzły jesteśmy wstanie równoważnie (nie zmieniając ''ciągłej struktury'')
przekształcić jeden z nich, aby otrzymać drugi. Zastanowimy się nad tym, jakie przekształcenia możemy uważać za równoważne. Na koniec sformułujemy i udowodnimy
twierdzenie Reidemeistera.
 
\subsection{Definicja}
 
Zaczniemy ten rozdział do sformuowania definicji węzła. Kiedy definiuje się matemayczne pojęcie ważne jest (przynajmniej takie wrażenie ma autor tej części pracy),
żeby definicja odpowiadała naszym oczekiwaniom, tj. żeby gwarantowała oczekiwane własności, a jednocześnie zachowywała możliwie jak największy
stopień ogólności.
 
Na pierwszy rzut oka następujące podejście wygląda dość rozsądnie.
 
 
\begin{definicja}
\label{zla_definicja}
 Węzłem nazywamy ciągłą funkcję $f\colon[0,1]\to\mathbb{R}^3$, która jest ciągła i spełnia następujące własności:
 \begin{enumerate}
  \item $f(0) = f(1)$,
  \item $\forall x,y\in(0,1) \ \ \ f(x) = f(y)\Rightarrow x = y$.
 \end{enumerate}
\end{definicja}
 
W rzeczywistości ma ono dość poważne wady.
Po pierwsze, powyższa definicja dopuszcza tak zwane dzikie węzły (rysunek poniżej), co jest sprzeczne z intuicyjnym pojmowaniem węzła, który może mieć
jedynie skończenie wiele pętli.
 
Po drugie wychodząc od definicji \ref{zla_definicja} mielibyśmy kłopot z określeniem tego, kiedy dane dwa węzły są równoważne. Naturalną definicją wydaje się
następująca: węzły $K$ i $J$ uważamy za równoważne, gdy istnieje rodzina węzłów $\lbrace K_t\rbrace_{t\in[0,1]}$, taka że $K_0 = K$, oraz $K_1 = J$. Oprócz tego
węzły $K_t$ i $K_s$ powinny leżeć dowolnie ''blisko'' dla $t$ dostatecznie bliskich $s$. W sensie tej definicji (nie wnikając w to, co oznacza,
że dwa węzły leżą ''blisko''), niewęzeł jest równoważny trójlistnikowi (patrz rysunek poniżej).
 
Rozwiązanie naszych problemów okazuje się zaskakująco proste. Zamiast definiować węzeł za pomocą ciągłej krzywej, moża posłużyć się łamaną. Skąd tem pomysł? Łamana ma skończenie wiele
wierzchołków, zatem powinniśmy bez specjalnych problemów wykluczyć z naszych rozważań tak zwane dzikie węzły. Przekształcenia równoważne ''powinny'' nie zmieniać relacji liniowej niezależności
pomiędzy wierzchołkami węzła (jak później się okaże), zatem nie powinniśmy się obawiać, że w jakiś złośliwy i nieoczekiwany sposób pojedyncza pętla okaże się równoważna niewęzłowi.
 
 
 
\begin{definicja}
 Odcinkiem domkniętym łączącym parę różnych punktów $p,q\in\mathbb{R}^3$ nazwiemy zbiór
 \begin{displaymath}
  [p,q] := \lbrace \lambda p + (1-\lambda)(q-p): \lambda\in[0,1]\rbrace.
 \end{displaymath}
\end{definicja}
 
\begin{definicja}
 Niech $p_1, p_2, \ldots, p_n\in\mathbb{R}^3$ będą parami różne. Łamaną o wierzchołkach $p_i$ nazwiemy zbiór
 \begin{displaymath}
  \bigcup_{i=1}^n [p_i, p_{i+1}].
 \end{displaymath}
 Łamaną tej postaci będziemy oznaczać przez $(p_1, p_2, \ldots, p_n)$. Gdy $p_1 = p_n$ mówimy o łamanej zamkniętej.
 Jeśli dodatkowo każdy punkt łamanej należy do dokładnie jednego odcinka (poza krańcami odcinków), mówimy wtedy o łamanej bez samoprzecięć. Odcinek $[p_i, p_{i+1}]$ będziemy dalej krótko oznaczać przez $I_i$.
\end{definicja}
 
\begin{definicja}
 Węzeł to łamana zamknięta w $\mathbb{R}^3$ bez samoprzecięć.
\end{definicja}
 
\textbf{Uwaga} Jeśli węzeł $K$ to łamana $(p_1, p_2, \ldots, p_n)$, wówczas zbiór $(p_{\sigma(1)}, p_{\sigma(2)},\ldots, p_{\sigma(n)})$, gdzie $\sigma$ to cykl długości $n$
definiuje ten sam węzeł $K$.
Co więcej, jeśli dla pewnego $i$ punkty $p_{i-1}, p_i, p_{i+1}$ są współliniowe, to oczywiście $K = (p_1, p_2, \ldots, p_{i-1}, p_{i+1}, \ldots, p_n)$. Z ostatniego spostrzeżenia łatwo
można wywnioskować, że węzeł jest jednoznacznie wyznaczony przez minimalny (w sensie inkluzji) zbiór wierzchołków łamanej, która ten węzeł definiuje.
 
\begin{definicja}
 Niech uporządkowany zbiór $(p_1, p_2,\ldots, p_n)$ definiuje węzeł $K$. Jeśli żaden jego właściwy podzbiór nie definiuje węzła $K$, wówczas elementy zbioru $\lbrace p_1, p_2, \ldots, p_n\rbrace$
 nazwiemy wierzchołkami węzła $K$.
\end{definicja}
 
Od tego momentu będziemy definiować węzeł poprzez łamane o minimalnej liczbie wierzchołków, chyba, że zostanie powiedziane inaczej. Na koniec tego podrozdziału podamy definicję splotu węzłów.
 
\begin{definicja}
 Splotem nazywamy sumę (mnogościową) skończenie wielu parami rozłącznych węzłów. Każdy jeden :) taki węzeł nazywamy składową splotu. W szcze gólmności węzeł jest splotem o jednej skłądowej.
 Nie-splotem nazwiemy sumę skończenie wielu rozłącznych niewęzłów (okręgów) leżących w jednej płaszczyźnie.
\end{definicja}
 
\textbf{Uwaga} W przypadku nie-splotu warunek leżenia w jednej płaszczyźnie jest istotny. Aby się o tym przekonać czytelnik może zechcieć odpowiedzieć na pytanie,
jak mogą leżeć względem siebie rozłączne okręgi w $\mathbb{R}^3$, a jak muszą w $\mathbb{R}^2$.
 
\subsection{Równoważność węzłów}
W tym podrozdziale zastanowimy się nad tym, kiedy dwa, pozornie różne węzły, w istocie możemy uważać za takie same (mimo iż explicite nie są tym samym węzłem).
Zaczniemy od zdefiniowania kilku pojęć.
 
\begin{definicja}
\label{elementarne_p}
 Niech dane będą węzły $J = (p_0, p_1, p_2, \ldots, p_n)$ oraz $K = (p_1, p_2, \ldots, p_n)$.
 Powiemy, że $J$ powstaje przez elementarnye przekształcenie węzła $K$ gdy:
 \begin{enumerate}
  \item punkt $p_0$ nie jest współliniowy z $p_{1}$ oraz z $p_{n}$,
  \item część wspólna trójkąta (brzegu trójkąta) o wierzchołkach w punktach $p_0, p_1, p_n$ z węzłem $K$ zawiera się w odcinku $[p_1, p_n]$.
 \end{enumerate}
 Przekształcenie odwrotne do elementarnego również jest przekształceniem elementarnym.
\end{definicja}
\textbf{Uwaga} Powyższą definicję można uogólnić zastępując węzeł $K$ węzłem $(p_0, p_1,\ldots, p_{i-1},p_{i+1}, \ldots, p_n)$ dla $i = 1,2,\ldots n-1$, oraz zastępując wierzchołki
$p_0, p_1, p_n$ wierzchołkami $p_i, p_{i-1}, p_{i+1}$ odpowiednio.
 
\begin{definicja}
 Mówimy, że węzły $J$ i $K$ są równoważne, gdy istnieje skończony ciąg węzłów $K_0, K_1, \ldots, K_n$, gdzie $K_0$ = $K$, $K_n = J$, oraz $K_{i+1}$ powstaje przez pewne elementarne
 przekształcenie węzła $K_i$.
\end{definicja}
 
Sprawdzenie, że podana relacja jest w istocie relacją równoważności pozostawiamy czytelnikowi jako proste ćwiczenie.
 
Dla przykładu rozważmy dowolny $n$-kąt wypukły na płaszczyźnie. Można łatwo pokazać przez indukcję (po liczbie wierzchołków), że każdy taki wielokąt jest niewęzłem. Istotnie, dla trójkąta
teza jest oczywista. Załóżmy jej prawdziwość dla wszystkich liczb naturalnych nie większych, niż $n$. Mając dowolny $n+1$-kąt wypukły $J = (k_0, k_1, k_2, \ldots, k_n)$ łatwo się przekonać,
że jest on elementarnym przekształceniem $n$-kąta $ K = (k_1, k_2, \ldots, k_n)$. To, że $K$ jest wielokątem wypukłym, podobnie jak to, że pierwszy warunek z definicji \ref{elementarne_p} jest spełniony, jest oczywiste.
Drugi warunek jest spełniony, ponieważ z wypukłości $K$ zachodzi $\left([p_0,p_1]\cup [p_0, p_n]\right) \cap K = \lbrace p_1, p_n\rbrace$.
 
Od tej chwili węzły równoważne będziemy uważać za tożsame, to znaczy zamiast pisać, że węzeł $J$ jest równoważny węzłowi $K$, będziemy pisać krótko $J=K$.
Odróżniać będziemy tylko te węzły, które nie są równoważne.
 
\subsection{Diagram}
Od początku tej pracy, z konieczności rysowaliśmy węzły (obiekty żyjące w przestrzeni trójwymiarowej) na płaszczyźnie.
W tym podrozdziale podamy ścisłą definicję tych dwuwymiarowych rysunków (diagramów) i pokażemy, że jeśli dwa węzły mają przedstawienie w postaci tego samego diagramu,
to są w istocie równoważne.
 
\begin{definicja}
 Rzutem zbioru $A\subseteq\mathbb{R}^3$ na płaszczyznę nazwiemy funkcję $p\colon A\to\mathbb{R}^2$ określoną wzorem $p(x,y,z) = (x,y)$. Kiedy $A$ jest węzłem, $p[A]$ nazywamy
 rzutem węzła $A$ na płaszczyznę.
\end{definicja}
 
 
\begin{definicja}
\label{rzut_reg}
 Rzutem regularnym węzła $K$ nazywamy takie rzutowanie $p$ węzła $K$ na płaszczyznę, że
 \begin{enumerate}
  \item dla każdego wierzchołka $v$, dla każdego $a\in K$ jeśli $p(v) = p(a)$, to $p=a$,
  \item dla każdych $a_1, a_2, a_3$ jeśli $p(a_1)=p(a_2)=p(a_3)$, to $a_1=a_2$ lub $a_1=a_3$ lub $a_2=a_3$.
 \end{enumerate}
\end{definicja}
 
Zakładająć, że węzeł ma regularny rzut na płaszczyznę, możemy w ścisły sposób zdefiniować pojęcie diagramu.
Diagramem węzła $K$ nazywamy następującą modyfikację regularnego rzutu $K$ na płaszczyznę. Jeśli przeciwobrazem punktu $(x_0, y_0)$ jest zbiór dwuelementowy
$\lbrace a,b\rbrace$, wówczas w tym punkcie przecinają się dwa różne odcinki, które są rzutami dwóch różnych odcinków węzła $K$, powiedzmy $I_a, I_b$.
Ponieważ rzut $K$ jest rzutem regularnym, więc jeden z tych odcinków leży pod drugim, co oznaczamy, jak na poniższym rysunku.
 
\begin{definicja}
 Mówimy, że dwa węzły $(p_i)$, $(q_j)$ są od siebie odległe o mniej, niż $t$, gdy mają tyle samo wierzchołków i gdy dla każdej pary wierzchołków zachodzi $d(p_k,q_k) < t$.
\end{definicja}
 
Udowodnimy teraz twierdzenie, które zagwarantuje nam, że dla każdego węzła można znaleźć węzeł dowolnie mu bliski, którego rzut na płaszczyznę jest regularny.
Będziemy do tego potrzebować następujących lematów.
 
\begin{lemat}
\label{LEM1}
 Dla dowolnego węzła $K = (v_1, \ldots, v_i, \ldots, v_n)$ i dowolnego $i = 1,2,\ldots n$ istnieje taki $\epsilon > 0$, że dla każdego $v'\in B(v_i, \epsilon)$
 węzły $K$ oraz $(v_1, \ldots, v', \ldots, v_n)$ są równoważne.
\end{lemat}
\begin{proof}
 
Ustalmy dowolne $i$. Dodawanie i odejmowanie indeksów w dalszej części dowodu należy rozumieć jak działanie modulo $n+1$. Możemy założyć, że punkty $v_{i-1}, v_i, v_{i+1}$ nie są
współliniowe. Zaczniemy dowód od skonstruowania dwóch stożków.
Stożek o wierzchołku $w$, środku podstawy $s$ i promieniu podstawy $r$ będę oznaczał $S(s,r,w)$.
 
Niech $p\in\mathbb{R}^3$ będzie wektorem długości jeden prostopadłym do odcinków $I_{i-1}$ oraz $I_i$. Oznaczmy przez $\alpha$ kąt pomiędzy odcinkami
$I_{i-2}$ i $I_{i-1}$.
Dobierzmy tak małą $\mu\in\mathbb{R}^3$, żeby kąt pomiędzy $I_{i-1}$ oraz $[v_{i-1}, v_i + \mu p]$ był mniejszy od $\alpha$. Możemy $\mu$ wyliczyć np. używając trygonometrii
w trójkącie prostokątnym o wierzchołkach $v_{i-1}, v_i, v_i + r\cdot p$, gdzie $r\in\mathbb{R}_+$. Połóżmy
\begin{displaymath}
 \epsilon = \frac{1}{2}\min\lbrace d(I_{i-1}, K\setminus(I_{i-2}\cup I_i)), \mu\rbrace, \hbox{ oczywiście }\epsilon>0.
\end{displaymath}
Wtedy
\begin{enumerate}
 \item $S(v_i, \epsilon, v_{i-1})\cap K \subseteq I_{i-1}\cup I_i$,
 \item dla każdej $\delta\in(0,\epsilon)$ zachodzi $[v_{i-1}, v_i+\delta p]\subseteq S(v_i, \epsilon, v_{i-1})$.
\end{enumerate}
Zauważmy, że $d(v_i, cl(K\setminus(I_{i-1}\cup I_i)) > 0$, więc (jako że $\mathbb{R}^3$ jest $T_4$) można znaleźć takie
$\gamma > 0$, że $B(v_i,\gamma)\cap cl(K\setminus(I_{i-1}\cup I_i)) = \emptyset$. Możemy wziąć takie $\gamma$, żeby $\gamma < \epsilon$.
 
Wtedy stożek
$S_{i-1} = S(v_i + \gamma(v_i-v_{i-1}), \epsilon, v_{i-1})$ spełnia własność $1$ oraz własność
\begin{displaymath}
 \hbox{dla każdego }v'\in B(v_i, \gamma) \ \ [v_{i-1}, v']\subseteq S_{i-1}.
\end{displaymath}
 
 
W analogiczny sposób konstruujemy stożek $S_i = S(v_i + \gamma'(v_i - v_{i+1}), \epsilon', v_{i+1})$. Zbiory $S_{i-1}$ i $S_i$ mają niepuste wnętrza, oraz
$v_i\in int(S_i)\cap int(S_{i+1})$. Istnieje więc taka liczba $r>0$, że $B(v_i, r)\subseteq S_i\cap S_{i+1}$.
 
Owe stożki pomogą nam pokazać, że dla każdego $v'\in B(v_i, r)$ węzeł $(v_1, \ldots, v', \ldots, v_n)$ jest równoważny $K$.
 
Ustalmy $v'\in B(v_i, r)$. Wtedy odcinki $[v_{i-1}, v'], [v',v_i], [v', v_{i+1}]$ należą do zbioru $S_i\cup S_{i-1}$, który jest rozłączny z $K\setminus(I_{i-1}\cup I_i)$.
Rozważmy dwa przypadki.
 
Jeżeli $v'\in I_i$ (gdy $v'\in I_{i-1}$, to postępujemy analogicznie), to węzeł $K$ można zdefiniować za pomocą łamanej $(v_1, \ldots, v_i, v', v_{i+1}, \ldots, v_n)$. Ponieważ
wnętrze odcinka $[v_{i-1}, v']$ jest rozłączne z $K$, więc $K$ jest równoważny węzłowi $(v_1, \ldots, v_{i-1}, v', v_{i+1}, \ldots, v_n)$.
 
Jeśli $v'\not\in I_i\cup I_{i-1}$, wówczas wnętrze odcinka $[v_i, v']$ jest rozłączne z $I_i\cup I_{i-1}$. Rozważmy odcinek $[v_{i-1}, v']$, jeśli przecina on odcinek $I_i$, to
wtedy odcinek $[v', v_i]$ nie przecina odcinka $I_{i-1}$, zatem bez zmniejszenia ogólności możemy założyć, że wnętrze odcinka $[v_{i-1}, v']$ jest rozłączne z $K$. Tworzymy
następujący ciąg
\begin{displaymath}
 K_0 = K, K_1 = (v_1,\ldots, v_{i-1}, v', v_i, \ldots, v_n), K_2 = (v_1,\ldots, v_{i-1}, v', \ldots, v_n).
\end{displaymath}
Czytelnik zechce się sam przekonać, że jest to ciąg przekształceń elementarnych.
Kładziemy $K' = K_2$ i kończymy dowód.
\end{proof}
\textbf{Uwaga} Ponieważ funkcja
 $ d(x,y)\mapsto d(p(x), p(y))$ jest nierosnąca, więc przy oznaczeniach, jak w dowodzie lematu zachodzi $p(v')\in B(p(v_i),\epsilon)$.
\begin{wniosek}
 Translacja węzła o odwolny wektor nie zmienia jego klasy równoważności.
\end{wniosek}
 
 
\begin{lemat}
 \label{LEM2}
 Rozpatrzmy węzeł $K = (v_1, \ldots, v_n)$. Niech $p$ oznacza rzut na płaszczyznę. Dla każdego $i$, dla każdego $\epsilon > 0$ istnieje węzeł
 węzeł $J = (q_1, \ldots, q_i, \ldots q_n)$ równoważny węzłowi $K$ odległy od $K$ o mniej niż $\epsilon$ oraz spełniający:
 dla dowolnego $r\in J$ i dla dowolnego $i$ zachodzi $p(q_i)\neq p(r)$, gdzie $p$ oznacza rzut prostokątny na płaszczyznę $OXY$.
\end{lemat}
 \begin{proof}
  Niech $i$ będzie dowolne.
  Niech $\epsilon'$ będzie dobrany do wierzchołka $v_i$, jak w tezie lematu \ref{LEM1}. W razie potrzeby możemy go zmniejszyć, tak by $\epsilon'<\epsilon$.
  Zauważmy, że $p[B(v_i, \epsilon')] = B(p(v_i),\epsilon')$, gdzie kula po lewej stronie równości jest z $\mathbb{R}^3$, a po prawej z $\mathbb{R}^2$. Ponieważ zbiór $p(K)$ jest nigdziegęstym podzbiorem płaszczyzny, więc możemy wybrać taki $q_i\in B(v_i, \epsilon')$,
  że $p[\lbrace q_i\rbrace]\cap p[K] = \emptyset$. Otrzymujemy w ten sposób nowy węzeł $(v_1, \ldots, q_i, \ldots, v_n)$. Powtarzamy opisaną procedurę $n-1$ razy i w efekcie
  otrzymujemy węzeł $J$. Ponieważ $\epsilon'$ był z lematu \ref{LEM1}, więc $J$ i $K$ są równoważne.
 \end{proof}
 
 
\begin{twierdzenie}
 Niech $K$ będzie węzłem o uporządkowanym zbiorze wierzchołków $(v_1, v_2, \ldots, v_n)$. Dla każdego $\epsilon > 0$ istnieje węzeł $K'$,
który jest odległy od węzła $K$ o nie więcej, niż $\epsilon$, oraz jego rzut na płaszczyznę $OXY$ jest regularny.
\end{twierdzenie}
\begin{proof}
 
Najpierw zastosujemy lemat \ref{LEM2} do węzła $K$ i powstały w ten sposób węzeł oznaczymy przez $K'$.
 
Ponieważ rzut węzła $K'$ spełnia pierwszy warunek z definicji \ref{rzut_reg}, więc dla dwóch różnych odcinków $I,J$ węzła
$K'$ zachodzi $|p(I)\cap p(J)| \le 1$.
 
Niech $\mathcal{A}$ będzie rodziną wszystkich takich podzbiorów zbioru odcinków węzła $K'$, że dla każdego $A\in\mathcal{A}$ zachodzi
\begin{enumerate}
 \item $|A| > 1$,
 \item istnieje taki $r\in\mathbb{R}^2$, że $\bigcap_{I\in A}p(I) = \lbrace r\rbrace$,
 \item dla każdego $J\not\in A$ $\left(\bigcap_{I\in A}p(I)\right)\cap p(J) = \emptyset$.
\end{enumerate}
 
Niech $a\in K'$ będzie punktem z wnętrza pewnego odcinka $I_i$, takim że istnieją różne od $a$ punkty $b, c\in K'$ takie że $p(a) = p(b) = p(c)$ i $b\neq c$.
Niech $\lbrace t_A\rbrace_{A\in\mathcal{A}}$ będzie ciągiem w $\mathbb{R}^2$, takim że $t_A = \bigcap_{I\in A}p(I)$. Niech $u\in\mathbb{R}^2$ będzie niezerowym wektorem prostopadłym do osi $OZ$, który
nie jest równoległy do $p(I_{i-1})$ oraz nie jest równoległy do $p(I_{i+1})$. Wtedy, używając tego samego argumentu, co w dowodzie lematu \ref{LEM2}, możemy znaleźć dostatecznie małą $\delta > 0$, że dla każdej $0 < \delta' < \delta$ istnieje wektor $w$, taki że
węzeł $(v_1, \ldots, v_{i-1}, v_i  + \delta'w, v_{i+1}  + \delta'w, v_{i+2}, \ldots, v_n)$ jest równoważny węzłowi $K'$, oraz rzut tego węzła spełnia punkt pierwszy
defnicji \ref{rzut_reg}. Ponieważ ciąg $t_A$ jest skończony, to
liczbę $\delta'$ można wziąć na tyle małą, żeby była mniejsza od dowolnej dodatniej odległości $d(p(I_j), t_A)$ dla $j \in \lbrace i-1, i+1\rbrace$.
Wtedy węzeł powstały z $K'$ przez zastąpienie $v_i, v_{i+1}$ wierzchołkami $v_i'=v_{i} + \delta'w, v_{i+1}'=v_{i+1}+\delta'w$
odpowiednio jest równoważny węzłowi $K'$, co więcej, dla każdego $r\in[v_{i-1}, v_i']\cup[v_i', v_{i+1}']\cup[v_{i+1}', v_{i+2}]$ istnieje co najwyżej jeden taki $r'\neq r$, że $p(r') = p(r)$.
Zauważmy, że po tej zastosowaniu tej procedury przynajmniej jeden element rodziny $\mathcal{A}$ ma jeden element mniej.
zatem w skończonej liczbie kroków dojdziemy do momentu, kiedy w rodzinie $\mathcal{A}$ zostaną jedynie zbiory dwuelementowe. Wtedy rzut otrzymanego węzła będzie regularny.
 
\end{proof}
 
 
Warto podkreślić, że dwa różne węzły (w sensie relacji równości) mogą mieć ten sam diagram, np. mając dany węzeł można przesunąć jego wierzchołek o największej trzeciej współrzędnej
o wersor równoległy do osi $OZ$. Okazuje się jednak, że jeśli dwa węzły mają ten sam diagram, to są równoważne.
 
 
\begin{twierdzenie}
 Jeśli dwa węzły $K = (v_1,\ldots, v_n)$ oraz $W = (w_1, \ldots, w_n)$ mają regularne rzutowanie oraz ich diagramy są równe, to $K$ i $J$ są równoważne.
\end{twierdzenie}
\begin{proof}
 Ustalmy wierzchołek $v_i$. Zauważmy, że bez straty ogólności możemy założyć, że rzut każdego z odcinków $I_i, I_{i-1}$ węzła $K$ zawiera co najwyżej jedno skrzyżowanie.
 Istotnie, jeśli
 rzut odcinka $I_i$ zawiera więcej skrzyżowań (oczywiście jest ich skończenie wiele), to niech $s_0$ oznacza skrzyżowanie najbliższe $p(v_i)$, a $s_1$ skrzyżowanie najbliższe
 punktowi $s_0$. Wybierzmy dowolny punkt z wnętrza odcinka $[s_0, s_1]$, oznaczmy go przez $s$. Niech $v' = p^{-1}(s)$. Wtedy łamana $(v_1, \ldots, v_i, v', v_{i+1})$ definiuje
 węzeł $K$. Wówczas odcinek $p[ [v_i, v']]$ zawiera dokładnie jedno skrzyżowanie. To samo możemy założyć o węźle $J$ dodając nowy wierzchołek $q'$ w punkcie $p^{-1}(s)$.
 
 Niech $I, J$ oznaczają te odcinki węzła $K$, że $p[I]\cap p[I_i]\neq\emptyset$ oraz $p[J]\cap p[I_{i-1}]\neq\emptyset$. Wtedy $I$ leży ,,pod'' odcinkiem $I_i$ i pod odcinkiem
 węzła $W$, który jest rzutowany na $p[I_i]$, albo ,,nad'' oboma tymi odcinkami. To samo się tyczy odcinka $J$. Stąd wynika, że wnętrza odcinków $[v_{i-1}, w_i], [w_i, v_{i+1}]$ są
 rozłączne z węzłem $K$. To, że $K\cap [v_i,w_i] = \lbrace v_i\rbrace$ jest oczywiste. Dlatego ciąg
 \begin{displaymath}
  K_0 = (v_1, \ldots, v_n), K_1 = (v_1, \ldots, v_{i-1}, w_i, v_i, v_{i+1}, \ldots, v_n), K_3 = (v_1, \ldots, v_{i-1}, w_i, v_{i+1}, \ldots, v_n)
 \end{displaymath}
jest ciągiem przekształceń elementarnych. Stosując powyżej opisaną procedurę do każdego wierzchołka węzła $K$ otrzymamy na końcu węzeł $J$, co znaczy, że te węzły są równoważne.
 
 \end{proof}
 
\subsection{Każdy kij ma dwa końce, czyli orientacja węzła}
Jak już wcześniej zostało powiedziane, węzeł to łamana w $\mathbb{R}^3$. Łamana z kolei jest wyznaczona jednoznacznie przez zbiór  swoich wierzchołków. Mając dany węzeł o wierzchołkach
$(v_1, v_2, \ldots, v_n)$, możemy go zorientować, to znaczy dla każdego odcinka węzła wybrać początek i koniec tego odcinka. Chcielibyśmy to zrobić w taki sposób, żeby każdy wierzchołek
był końcem dokładnie jednego odcinka. W tym sensie węzeł $(v_1, v_2, \ldots, v_n)$ jest węzłem różnym od węzła $(v_n, \ldots, v_2, v_1)$, mimo, że jako podzbiory $\mathbb{R}^3$ są
sobie równe. Teraz spróbujemy sformalizować to, co dotychczas powiedzieliśmy.
 
\begin{definicja}
 Odcinkiem zorientowanym w $\mathbb{R}^3$ o początku w $p$ i końcu w $q$ (zakładamy, że $p\neq q$) nazywamy liniowe włożenie $l\colon[0,1]\to\mathbb{R}^3$, takie że $l(0) = p, l(1) = q$ i oznaczamy przez
 $[p,q]_o$.
\end{definicja}
 
Innymi słowy odcinek zorientowany to odcinek z wyróżnionym początkiem i końcem. Co więcej, przy oznaczeniach z definicji zachodzi
\begin{displaymath}
 q-p =  |q-p|\cdot\frac{l'(x)}{|l'(x)|}, \hbox{ dla dowolnego } 0 < x < 1.
\end{displaymath}
 
\begin{definicja}
 Orientacją węzła $K = (v_1, v_2, \ldots, v_n)$ nazywamy takie zorientowanie jego odcinków, że każdy wierzchołek jest końcem (równoważnie - początkiem) dokładnie jednego odcinka.
\end{definicja}
 
Nie trudno się przekonać, że każdy węzeł można zorientować na dokładnie dwa sposoby. Istotnie, wybranie orientacji jednego odcinka węzła jednoznacznie wyznacza orientacje całego
węzła.
 
Przyjmiemy następującą konwencję, jeśli powiemy, że węzeł $K = (v_1, v_2, \ldots, v_n)$ jest węzłem zorientowanym, będziemy przez to rozumieć, że $I_1 = [v_1, v_2]_o$.
 
\textbf{Uwaga} Niech będzie dany węzeł $K = (v_1, v_2, \ldots, v_n)$, niech ponadto uporządkowane $n$-tki $K_i = (v_{i_1}, v_{i_2}, \ldots, v_{i_n})$
oraz $K_j = (v_{j_1}, v_{j_2}, \ldots, v_{j_n})$ definiują ten sam węzeł $K$. Niech $O_i, O_j$ oznaczają orientację zorientowanych węzłów $K_i, K_j$ odpowiednio. Powiemy, że orientacja
$O_i$ jest równoważna orientacji $O_j$, gdy isnieje taka $\sigma\in S_n$, że dla każdego $k$ zachodzi $\sigma(i_k) = j_k$. Sprawdzenie, że ta relacja jest w istocie relacją równoważności
pozostawiamy jako proste ćwiczenie.
 
\begin{definicja}
 Węzłem zorientowanym nazywamy węzeł z wybraną orientacją.
\end{definicja}
 
Przekształcenia elementarne węzła zorientowanego definiuje się w sposób oczywisty. Do zestawu punktów $(1), (2)$ z definicji \ref{elementarne_p} należy dodać następujący warunek.
\begin{displaymath}
 \hbox{Odcinki zorientowane koincydentne z wierzchołkiem } p_0 \hbox{ są równe }[p_n, p_0]_o, [p_0, p_1]_o.
\end{displaymath}
 
\begin{definicja}
 Węzły zorientowane nazywamy równoważnie zorientowanymi, gdy w skończenie wielu krokach za pomocą przekształceń elementarnych jednego węzła zorientowanego
 możemy otrzymać drugi.
\end{definicja}
Należy wyraźnie powiedzieć, że równoważność dwóch węzłów jest istotnie inną relacją, niż równoważność węzłów w sensie orientacji. Istnieją bowiem przykłady węzłów równoważnych
ale nie równoważnych w sensie orientacji. Opisanie ich dokładnie wykracza jednak poza możliwości intelektualne autora, dlatego fakt ten, z wyższej konieczności, jest podany
czysto informacyjnie.
Na koniec tego rozdziału podamy jeszcze jedną definicję.
\begin{definicja}
 Odwrotnością węzła zorientowanego $K = (v_1, v_2, \ldots, v_n)$ nazywamy węzeł $K^r = (v_n, v_{n-1}, \ldots, v_1)$. Powiemy, że $K$ jest węzłem odwracalnym, jeżeli $K$ i $K^r$ są
 równoważne w sensie orientacji. Dla węzła $J$ niezorientowanego, powiemy że jest on odwracalny, jeśli istnieje taka orientacja węzła $J$, że jest on odwracalny (jako węzeł
 zorientowany z tą wybraną orientacją).
\end{definicja}
 
 
\subsection{Ruchy Reidemeistera}
W tym podrozdziale przedstawimy pierwsze poważne narzędzie, które w wielu przypadkach pozwoli nam rozstrzygnąć, czy dwa węzły są równoważne, czy też nie.
Metody komblinatoryczne, ble, ble, ble...
 
Wcześniej pokazaliśmy, że problem równoważności węzłów można próbować rozstrzygać posługując się diagramami tych węzłów. Sformułujemy i udowodnimy twierdzenie wiążące...
 
\begin{definicja}
 Dwa diagramy uważa się za równoważne, gdy wykonując skończoną liczbę przekształceń zwanych ruchami Reidemeister'a, można
 z jednego diagramu otrzymać drugi
\end{definicja}
 
 
\begin{definicja}
 Ruchy Reidemeistera - rysunki.
\end{definicja}
 
Trzy ruchy wraz z ich odwrotnościami.
 
\begin{twierdzenie}{(Reidemeister'a)}
Dwa węzły są równoważne, wtedy i tylko wtedy, gdy ich diagramy są równoważne.
\end{twierdzenie}
 
Trochę pozachwycam się tym twierdzeniem i prostotą dowodu, uzasadnię, czemu przyjąłem taką definicję i wystarczy...