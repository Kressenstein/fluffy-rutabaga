\documentclass[a4paper, fleqn]{extarticle}
\usepackage{geometry}
\usepackage{Alegreya, euler}
\usepackage{microtype}

%\usepackage{parskip}
\usepackage{amssymb, amsmath, amsthm}
\usepackage{lipsum}
\usepackage{verbatim}

\newcommand{\C}{\mathbb C}
\newcommand{\R}{\mathbb R}
\newcommand{\Q}{\mathbb Q}
\newcommand{\Z}{\mathbb Z}
\newcommand{\N}{\mathbb N}
\newcommand{\pies}{\mathcal{P}}
\newcommand{\kot}{\mathcal{K}}

\input{main_tikz}

\newcounter{dummy}
\numberwithin{dummy}{section}
\newtheorem{definicja}[dummy]{Definicja}
\newtheorem{twierdzenie}[dummy]{Twierdzenie}
\newtheorem{lemat}[dummy]{Lemat}
\newtheorem{wniosek}[dummy]{Wniosek}

\usepackage[polish]{babel}
\usepackage[utf8]{inputenc}
\usepackage[T1]{fontenc}
\selectlanguage{polish}

\title{Teoria węzłów}
\author{nasze nazwiska}

\begin{document}
\maketitle
\tableofcontents

% Maciek
\section{Definicja węzła}

Dzień dobry! :)

Piszę, próbuję i zamykam, hahaha! :)

% Szymon
\section{Grupa kolorująca}
Tekst.
\input{3/chapter_3}
% Remigiusz
\newpage

\section{Wielomian Jonesa}

Przed wprowadzeniem kolejnego wielomianowego niezmiennika przyjrzymy się ich historii.
Znamy już wielomian J. Alexandera, który został odkryty około roku 1928.
W 1969 J. Conway znalazł sposób na wyznaczenie wielomianu Alexandera dla dowolnego splotu przy użyciu tak zwanej relacji kłębiastej\footnote{skein relation}.
Jest to równanie wiążące wielomian splotu z wielomianami splotów o zmienionym jednym skrzyżowaniu w diagramie splotu wyjściowego.
Relacja kłębiasta okazała się kluczem do sukcesu.

Vaughan Jones, matematyk nowozelandzki, odkrył w 1984 nowy wielomian dla splotów jako produkt uboczny podczas pracy nad algebrami operatorowymi.
Odkrycie Jonesa było przełomowe, a już cztery miesiące później ogłoszono znalezienie nowego niezmiennika: wielomianu HOMFLY, którego nazwa pochodzi od pierwszych liter nazwisk odkrywców, to jest: Hoste, Ocneanu, Millett, Freyd, Lickorish, Yetter.

Aby lepiej zrozumieć wielomian Jonesa przyjrzymy się najpierw prostszej konstrukcji, nawiasowi Kauffmana.
Później zajmiemy się węzłami alternującymi.
% W tej sekcji zbadamy inny wielomianowy niezmiennik węzłów, wielomian Jonesa.
% Został odkryty w 1984 roku przez Vaughana Jonesa, znajduje zastosowanie między innymi przy badaniu węzłów przemiennych.

\subsection{Nawias Kauffmana}
Zaczniemy od zdefiniowania nawiasu Kauffmana.
Przypomnijmy, wielomian Laurenta zmiennej $X$ to formalny symbol $f=a_r X^r + \ldots + a_s X^s$, gdzie $r, s, a_r, \ldots, a_s$ są całkowite i $r \le s$.

Poszukujemy niezmiennika dla splotów o kilku prostych własnościach.
Przede wszystkim żądamy, by niewęzłowi przypisany był wielomian $1$: $\langle \NieWezel \rangle = 1$.
Po drugie chcemy móc wyznaczać nawiasy znając je dla prostszych splotów, co zapiszemy symbolicznie $\langle \PrawyKrzyz \rangle = A \langle \PrawyGladki \rangle + B \langle \LewyGladki \rangle$.
Zależy nam wreszcie na tym, by móc dodać do splotu trywialną składową: $\langle L \cup \NieWezel \rangle = C \langle L \rangle$.

Prosty rachunek pokazuje wpływ drugiego ruchu Reidemeistera na nawias:
\[
	\langle \begin{tikzpicture} [scale=0.025,baseline=-3]
			\path[TEXTARC] (0,10) .. controls (10,5) and (10,-9) .. (0,-14);
			\path[TEXTARC] (10,10) .. controls (8,9) .. (7,8);
			\path[TEXTARC] (3,4.5) .. controls (-1,0) and (-1,-4) .. (3,-8);
			\path[TEXTARC] (10,-14) .. controls (8,-13) .. (7,-12);
	\end{tikzpicture} \rangle = (A^2 + ABC + B^2) \langle \LewyGladki \rangle + BA \langle \PrawyGladki \rangle\stackrel{?}{=} \langle \PrawyGladki \rangle.
\]

Aby zachodziła ostatnia równość wystarczy (chociaż wcale nie trzeba) przyjąć $B = A^{-1}$, co wymusza na nas $C = -A^2 - A^{-2}$.
W ten sposób odkryliśmy następującą definicję.

\begin{definicja}
	\emph{Nawias Kauffmana} $\langle D \rangle$ dla diagramu splotu $D$ to wielomian Laurenta zmiennej $A$, który jest niezmienniczy ze względu na gładkie deformacje diagramu, a przy tym spełnia trzy aksjomaty:
	\begin{enumerate}
		\item $\langle \NieWezel \rangle=1$
		\item $\langle D \sqcup \NieWezel \rangle = (-A^{-2} - A^2) \langle D \rangle$
		\item $\langle \PrawyKrzyz \rangle = A \langle \PrawyGladki \rangle + A^{-1} \langle \LewyGladki  \rangle$
	\end{enumerate}
\end{definicja}

Tutaj $\NieWezel$ oznacza standardowy diagram dla niewęzła, $D \sqcup \NieWezel$ jest diagramem, który powstaje z $D$ przez dodanie nieprzecinającej go krzywej zamkniętej, zaś trzy symbole $\PrawyKrzyz$, $\PrawyGladki$ oraz $\LewyGladki $ odnoszą się do diagramów, które są identyczne wszędzie poza małym obszarem.
Diagramy $\PrawyGladki$ oraz $\LewyGladki$ nazywa się odpowiednio dodatnim (prawym) i ujemnym (lewym) wygładzeniem $\PrawyKrzyz$

\begin{lemat}
	Nawias Kauffmana dowolnego diagramu można wyznaczyć w skończonie wielu krokach.
\end{lemat}

\begin{proof}
	Jeżeli diagram $D$ ma $n$ skrzyżowań, to nieustanne stosowanie aksjomatu trzeciego pozwala na zapisanie $\langle D \rangle$ jako sumy $2^n$ składników, z których każdy jest po prostu zamkniętą krzywą i ma trywialny nawias ($\langle \MalyNieWezel \rangle = 1$).
	Nawias sumy wyznacza się korzystając z drugiego aksjomatu.
\end{proof}

Przedstawimy teraz wpływ ruchów Reidemeistera na nawias Kauffmana.

\begin{lemat}
	Pierwszy ruch Reidemeistera zmienia nawias Kauffmana zgodnie z poniższą regułą.
	Pozosałe ruchy Reidemeistera nie zmieniają nawiasu.
	\[
		% pierwszy ruch Reidemeistera
		\left\langle\begin{tikzpicture}[scale=0.05, baseline=-6]
			\clip (-12,-12) rectangle (1,7);
			\path[ARC] (-10,7) .. controls (-10,3) and (-10,0) .. (-6,-4);
			\path[ARC] (-6,0) .. controls (2,8) and (2,-10) .. (-6,-4);
			\path[ARC] (-10,-11) .. controls (-10,-8) and (-10,-5) .. (-9,-4);
		\end{tikzpicture}\right\rangle
		= -A^{-3}
		\left\langle\ \begin{tikzpicture}[scale=0.05,baseline=-6]
			\path[ARC] (-10,7) -- (-10,-11);
		\end{tikzpicture}\ \right\rangle
		\,\bullet\,
		% drugi ruch Reidemeistera
		\left\langle\begin{tikzpicture} [scale=0.04,baseline=-5]
			\path[ARC](0,10) .. controls (10,5) and (10,-9) .. (0,-14);
			\path[ARC] (10,10) .. controls (8,9) .. (7,8);
			\path[ARC] (3,4.5) .. controls (-1,0) and (-1,-4) .. (3,-8);
			\path[ARC] (10,-14) .. controls (8,-13) .. (7,-12);
		\end{tikzpicture}\right\rangle
		=
		\left\langle\ \begin{tikzpicture} [scale=0.04, baseline=-5]
		\path[ARC] (0,10) .. controls (3,8) and (3,0) .. (3,-2) .. controls (3,-4) and (3,-12) .. (0,-14);
		\path[ARC] (10,10) .. controls (7,8) and (7,0) .. (7,-2) .. controls (7,-4) and (7,-12) .. (10,-14);
		\end{tikzpicture}\ \right\rangle
		\,\bullet\,
		% trzeci ruch Reidemeistera
		\left\langle\begin{tikzpicture} [scale=0.04, auto, baseline=-6]
			\path[ARC] (-10,10) -- (-6.6,6);
			\path[ARC] (-4,3) -- (10,-14);
			\path[ARC] (10,10) -- (6.6,6);
			\path[ARC] (4,3) -- (1.6,0);
			\path[ARC] (-1.6,-4) -- (-10,-14);
			\path[ARC] (-14,-2) .. controls (-6, -2) and (-6,8) .. (0,8);
			\path[ARC] (14,-2) .. controls (6, -2) and (6,8) .. (0,8);
		\end{tikzpicture}\right\rangle
		=
		\left\langle\begin{tikzpicture} [scale=0.04, auto, baseline=-6]
			\begin{scope}[xshift=1300,rotate=180,yshift=110]
				\path[ARC] (-10,10) -- (-6.6,6);
				\path[ARC] (-4,3) -- (10,-14);
				\path[ARC] (10,10) -- (6.6,6);
				\path[ARC] (4,3) -- (1.6,0);
				\path[ARC] (-1.6,-4) -- (-10,-14);
				\path[ARC] (-14,-2) .. controls (-6, -2) and (-6,8) .. (0,8);
				\path[ARC] (14,-2) .. controls (6, -2) and (6,8) .. (0,8);
			\end{scope}
		\end{tikzpicture}\right\rangle.
	\]
\end{lemat}

\begin{proof}
Pierwszy ruch Reidemeistera:
\[
\left\langle\begin{tikzpicture}[scale=0.05, baseline=-6]
	\clip (-12,-12) rectangle (1,7);
	\path[ARC] (-10,7) .. controls (-10,3) and (-10,0) .. (-6,-4);
	\path[ARC] (-6,0) .. controls (2,8) and (2,-10) .. (-6,-4);
	\path[ARC] (-10,-11) .. controls (-10,-8) and (-10,-5) .. (-9,-4);
\end{tikzpicture}\right\rangle
\stackrel{K3}{=}
A\left\langle\begin{tikzpicture}[scale=0.05, baseline=-6]
	\clip (-12,-12) rectangle (1,7);
	\path[ARC] (-10,7) .. controls (-10,3) and (-9,-3) .. (-6,0);
	\path[ARC] (-6,0) .. controls (2,8) and (2,-10) .. (-6,-4);
	\path[ARC] (-10,-11) .. controls (-10,-8) and (-10,-1) .. (-6,-4);
\end{tikzpicture}\right\rangle
+A^{-1}\left\langle
\begin{tikzpicture}[scale=0.05, baseline=-6] 
	\clip (-12,-12) rectangle (1,7);
	\path[ARC] (-10,7) .. controls (-10,3) and (-7,-2) .. (-9,-4);
	\path[ARC] (-5,1) .. controls (2,8) and (2,-10) .. (-5,-5);
	\path[ARC] (-5,1) .. controls (-6.5,-0.5) and (-6.5,-3.5) .. (-5,-5);
	\path[ARC] (-10,-11) .. controls (-10,-8) and (-10,-5) .. (-9,-4);
\end{tikzpicture}\right\rangle
\stackrel{K2}{=}
A\left\langle\ 
\begin{tikzpicture}[scale=0.05, baseline=-6]
	\path[ARC] (-10,7) -- (-10,-11);
\end{tikzpicture}\ 
\right\rangle
+A^{-1}(-A^{-2}-A^2)\left\langle\ 
\begin{tikzpicture}[scale=0.05, baseline=-6] 
	\path[ARC] (-10,7) -- (-10,-11);
\end{tikzpicture}
\ \right\rangle
=
-A^{-3}\left\langle\ 
\begin{tikzpicture}[scale=0.05, baseline=-6] 
	\path[ARC] (-10,7) -- (-10,-11);
\end{tikzpicture}\ 
\right\rangle
\]
Pierwsza równość wynika z $K3$, druga z $K2$, trzecia jest oczywista.
Dla drugiego ruchu:
\begin{eqnarray*}
\left\langle\begin{tikzpicture} [scale=0.04,baseline=-5] 
	\path[ARC](0,10) .. controls (10,5) and (10,-9) .. (0,-14);
	\path[ARC] (10,10) .. controls (8,9) .. (7,8);
	\path[ARC] (3,4.5) .. controls (-1,0) and (-1,-4) .. (3,-8);
	\path[ARC] (10,-14) .. controls (8,-13) .. (7,-12);
\end{tikzpicture}\right\rangle
&\stackrel{K3}{=}&
\phantom{-}A^{\phantom{-2}}
\left\langle\begin{tikzpicture} [scale=0.04,baseline=-5] 
	\clip (-2,-14) rectangle (12,10);
	\path[ARC] (0,10) .. controls (4.5,6) and (5.5,6) .. (10,10);
	\path[ARC] (0,-14) .. controls (4,-11) .. (6,-8) .. controls (12,0) and (-5,0) .. (3,-8);
	\path[ARC] (10,-14) .. controls (8.5,-13) .. (7,-12);
\end{tikzpicture}\right\rangle
+
\phantom{A}A^{-1}
\left\langle\begin{tikzpicture} [scale=0.04,baseline=-5]
	\clip (-2,-14) rectangle (12,10);
	\path[ARC] (10,10) .. controls (8,9) .. (7,8);
	\path[ARC] (10,-14) .. controls (8,-13) .. (7,-12);
	\path[ARC](10,10) .. controls (0,5) and (15,-4) .. (0,-14);	
	\path[ARC] (0,10) .. controls (10,5) and (-3,0) .. (3,-8);
\end{tikzpicture}\right\rangle
\stackrel{K1}{=}
-A^{-2}
\left\langle\begin{tikzpicture} [scale=0.04, baseline=-5]
	\clip (-2,-14) rectangle (12,10);
	\path[ARC] (0,10) .. controls (4.5,6) and (5.5,6) .. (10,10);
	\path[ARC] (0,-14) .. controls (4.5,-10) and (5.5,-10) .. (10,-14);
\end{tikzpicture}\right\rangle
+
\phantom{A}
A^{-1}
\left\langle\begin{tikzpicture} [scale=0.04,baseline=-5] 
	\clip (-2,-14) rectangle (12,10);
	\path[ARC] (10,10) .. controls (8,9) .. (7,8);
	\path[ARC] (10,-14) .. controls (8,-13) .. (7,-12);
	\path[ARC](10,10) .. controls (0,5) and (15,-4) .. (0,-14);	
	\path[ARC] (0,10) .. controls (10,5) and (-3,0) .. (3,-8);
\end{tikzpicture}\right\rangle
\\
&\stackrel{K3}{=}&
-A^{-2}
\left\langle\begin{tikzpicture} [scale=0.04, baseline=-5]
	\clip (-2,-14) rectangle (12,10);
	\path[ARC] (0,10) .. controls (4.5,6) and (5.5,6) .. (10,10);
	\path[ARC] (0,-14) .. controls (4.5,-10) and (5.5,-10) .. (10,-14);
\end{tikzpicture}\right\rangle
+A^{-1}A
\left\langle\begin{tikzpicture} [scale=0.04, baseline=-5] 
	\clip (-2,-14) rectangle (12,10);
	\path[ARC] (0,10) .. controls (3,8) and (3,0) .. (3,-2) .. controls (3,-4) and (3,-12) .. (0,-14);
	\path[ARC] (10,10) .. controls (7,8) and (7,0) .. (7,-2) .. controls (7,-4) and (7,-12) .. (10,-14);
\end{tikzpicture}\right\rangle
+
A^{-1}A^{-1}
\left\langle\begin{tikzpicture} [scale=0.04, baseline=-5]
	\clip (-2,-14) rectangle (12,10);
	\path[ARC] (0,10) .. controls (4.5,6) and (5.5,6) .. (10,10);
	\path[ARC] (0,-14) .. controls (4.5,-10) and (5.5,-10) .. (10,-14);
\end{tikzpicture}\right\rangle
=%\phantom{-A^{-2}}
\left\langle\begin{tikzpicture} [scale=0.04, baseline=-5]
	\clip (-2,-14) rectangle (12,10);
	\path[ARC] (0,10) .. controls (3,8) and (3,0) .. (3,-2) .. controls (3,-4) and (3,-12) .. (0,-14);
	\path[ARC] (10,10) .. controls (7,8) and (7,0) .. (7,-2) .. controls (7,-4) and (7,-12) .. (10,-14);
\end{tikzpicture}\right\rangle
\end{eqnarray*}
Dla trzeciego ruchu:
\begin{eqnarray*}
\left\langle\begin{tikzpicture} [scale=0.04, auto, baseline=-6] 
	\path[ARC] (-10,10) -- (-6.6,6);
	\path[ARC] (-4,3) -- (10,-14);
	\path[ARC] (10,10) -- (6.6,6);
	\path[ARC] (4,3) -- (1.6,0);
	\path[ARC] (-1.6,-4) -- (-10,-14);
	\path[ARC] (-14,-2) .. controls (-6, -2) and (-6,8) .. (0,8);
	\path[ARC] (14,-2) .. controls (6, -2) and (6,8) .. (0,8);
\end{tikzpicture}\right\rangle
&\stackrel{K3}{=}&
A
\left\langle\begin{tikzpicture} [scale=0.04, auto, baseline=-6] 
	\path[ARC] (-10,10) -- (-6.6,6);
	\path[ARC] (10,10) -- (6.6,6);
	\path[ARC] (-14,-2) .. controls (-6, -2) and (-6,8) .. (0,8);
	\path[ARC] (14,-2) .. controls (6, -2) and (6,8) .. (0,8);
	\path[ARC] (-10,-14) .. controls (0,-3) and (0,-3) .. (10,-14);
	\path[ARC] (-4,3) .. controls (0,-2) .. (4,3);
\end{tikzpicture}\right\rangle
+A^{-1}
\left\langle\begin{tikzpicture} [scale=0.04, auto, baseline=-6]
	\path[ARC] (-14,-2) -- (14,-2);
	\path[ARC] (-10,10) .. controls (-7,6) and (-4,4) .. (-4,0);
	\path[ARC] (-10,-14) .. controls (-7,-10) and (-4,-8) .. (-4,-4);
	\path[ARC] (10,10) .. controls (7,6) and (4,4) .. (4,0);
	\path[ARC] (10,-14) .. controls (7,-10) and (4,-8) .. (4,-4);
\end{tikzpicture}\right\rangle
\stackrel{R2}{=}
A
\left\langle\begin{tikzpicture} [scale=0.04, auto, baseline=-6]
	\path[ARC] (-14,-2) -- (14,-2);
	\path[ARC] (-10,-14) .. controls (0,-3) and (0,-3) .. (10,-14);
	\path[ARC] (-10,10) .. controls (0,-1) and (0,-1) .. (10,10);
\end{tikzpicture}\right\rangle
+A^{-1}
\left\langle\begin{tikzpicture} [scale=0.04, auto, baseline=-6] 
	\path[ARC] (-14,-2) -- (14,-2);
	\path[ARC] (-10,10) .. controls (-7,6) and (-4,4) .. (-4,0);
	\path[ARC] (-10,-14) .. controls (-7,-10) and (-4,-8) .. (-4,-4);
	\path[ARC] (10,10) .. controls (7,6) and (4,4) .. (4,0);
	\path[ARC] (10,-14) .. controls (7,-10) and (4,-8) .. (4,-4);
\end{tikzpicture}\right\rangle
\\
&\stackrel{R2}{=}&
A
\left\langle\begin{tikzpicture} [scale=0.04, auto, baseline=-6,yscale=-1] 
	\path[ARC] (-10,10) -- (-6.6,6);
	\path[ARC] (10,10) -- (6.6,6);
	\path[ARC] (-14,-2) .. controls (-6, -2) and (-6,8) .. (0,8);
	\path[ARC] (14,-2) .. controls (6, -2) and (6,8) .. (0,8);
	\path[ARC] (-10,-14) .. controls (0,-3) and (0,-3) .. (10,-14);
	\path[ARC] (-4,3) .. controls (0,-2) .. (4,3);
\end{tikzpicture}\right\rangle
+A^{-1}
\left\langle\begin{tikzpicture} [scale=0.04, auto, baseline=-6] 
	\path[ARC] (-14,-2) -- (14,-2);
	\path[ARC] (-10,10) .. controls (-7,6) and (-4,4) .. (-4,0);
	\path[ARC] (-10,-14) .. controls (-7,-10) and (-4,-8) .. (-4,-4);
	\path[ARC] (10,10) .. controls (7,6) and (4,4) .. (4,0);
	\path[ARC] (10,-14) .. controls (7,-10) and (4,-8) .. (4,-4);
\end{tikzpicture}\right\rangle
\stackrel{K3}{=}
\left\langle\begin{tikzpicture} [scale=0.04, auto, baseline=-6] 
\begin{scope}[xshift=1300,rotate=180,yshift=110]
	\path[ARC] (-10,10) -- (-6.6,6);
	\path[ARC] (-4,3) -- (10,-14);
	\path[ARC] (10,10) -- (6.6,6);
	\path[ARC] (4,3) -- (1.6,0);
	\path[ARC] (-1.6,-4) -- (-10,-14);
	\path[ARC] (-14,-2) .. controls (-6, -2) and (-6,8) .. (0,8);
	\path[ARC] (14,-2) .. controls (6, -2) and (6,8) .. (0,8);
\end{scope}
\end{tikzpicture}\right\rangle
\end{eqnarray*}
korzystaliśmy tu z własności drugiego ruchu.
\end{proof}

Okazało się, że użycie najprostszego, I ruchu Reidemeistera, ,,psuje'' nawias!
W akcie desperacji moglibyśmy zmienić definicję, zaniechamy tego i przejdziemy do kolejnego składnika w przepisie na wielomian Jonesa.

\subsection{Spin}
Przypomnijmy, że znak skrzyżowania na diagramie to liczba $1$ lub $-1$:
$\operatorname{sign}
	\begin{tikzpicture}[scale=0.03, baseline=-3]
	\path[TEXTARC,->] (-5,-5) -- (5,5);
	\path[TEXTARC] (5,-5) -- (1.5,-1.5);
	\path[TEXTARC,<-] (-5,5) -- (-1.5,1.5);
	\end{tikzpicture}
 = +1$,
$\operatorname{sign} \begin{tikzpicture}[scale=0.03, baseline=-3]
\path[TEXTARC,->] (5,-5) -- (-5,5);
\path[TEXTARC] (-5,-5) -- (-1.5,-1.5);
\path[TEXTARC,<-] (5,5) -- (1.5,1.5);
\end{tikzpicture} = -1$.

\begin{definicja}
	Niech $D$ będzie diagramem zorientowanego splotu lub węzła.
	\textbf{Spinem} $D$ jest $w(D) = \sum_c \operatorname{sign} c$, gdzie sumowanie przebiega po wszystkich skrzyżowaniach.
\end{definicja}

\begin{przyklad}
Spinem trójlistnika w takiej wersji jest $+3$:
\[
	\begin{tikzpicture}[scale=0.035]
		\clip (-16,-16) rectangle (16,17); %nie etykietuj
		\foreach \x in {270,30, 150}
		\path[TEXTARC,->-] (15+\x:6) .. controls (130+\x:25) and (200+\x:25) .. (225+\x:10);
		\node[red] (C1) at (30:14) {\small $+1$};
		\node[red] (C2) at (150:14) {\small $+1$};
		\node[red] (C3) at (270:14) {\small $+1$};
	\end{tikzpicture}
\]
\end{przyklad}

\begin{lemat}
Tylko I ruch Reidemeistera zmienia spin:
$
w(\begin{tikzpicture}[scale=0.025, baseline=-3]
	\clip (-12,-12) rectangle (1,7);
	\path[TEXTARC] (-10,7) .. controls (-10,3) and (-10,0) .. (-6,-4);
	\path[TEXTARC] (-6,0) .. controls (2,8) and (2,-10) .. (-6,-4);
	\path[TEXTARC] (-10,-11) .. controls (-10,-8) and (-10,-5) .. (-9,-4);
\end{tikzpicture})
=
w(\ \begin{tikzpicture}[scale=0.025,baseline=-4]
	\path[TEXTARC] (-10,7) -- (-10,-11);
\end{tikzpicture}\ )-1
$, pozostałe nie mają wpływu.
Spin nie zależy od orientacji.
\end{lemat}

\begin{proof}
Proste ćwiczenie.
\end{proof}

\subsection{Wielomian Jonesa}

\begin{definicja}
\emph{Wielomian Jonesa} zorientowanego splotu to wielomian Laurenta $V(L)\in\Z[t^{1/2},t^{-1/2}]$ określony przez
\[V(L)=\left[ (-A)^{-3w(D)}\langle D\rangle\right]_{t^{1/2}=A^{-2}},\]
gdzie $D$ to dowolny diagram dla $L$.
\end{definicja}

\begin{twierdzenie}
Wielomian Jonesa jest niezmiennikiem zorientowanych splotów.
\end{twierdzenie}

\begin{proof}
%Skorzystamy z tego, że indeks zaczepienia jest niezmiennikiem.
Wystarczy pokazać niezmienniczość $(-A)^{-3w(D)}\langle D\rangle$ na ruchy Reidemeistera.
ale
\begin{eqnarray*}
(-A)^{-3
w\left(\begin{tikzpicture}[scale=0.025, baseline=-3]
	\clip (-12,-12) rectangle (1,7);
	\path[TEXTARC] (-10,7) .. controls (-10,3) and (-10,0) .. (-6,-4);
	\path[TEXTARC] (-6,0) .. controls (2,8) and (2,-10) .. (-6,-4);
	\path[TEXTARC] (-10,-11) .. controls (-10,-8) and (-10,-5) .. (-9,-4);
\end{tikzpicture}\right)}
\left\langle\begin{tikzpicture}[scale=0.025, baseline=-3]
	\clip (-12,-12) rectangle (1,7);
	\path[TEXTARC] (-10,7) .. controls (-10,3) and (-10,0) .. (-6,-4);
	\path[TEXTARC] (-6,0) .. controls (2,8) and (2,-10) .. (-6,-4);
	\path[TEXTARC] (-10,-11) .. controls (-10,-8) and (-10,-5) .. (-9,-4);
\end{tikzpicture}\right\rangle
&=&
(-A)^{-3
w\left(\ \begin{tikzpicture}[scale=0.025,baseline=-4]
	\path[TEXTARC] (-10,7) -- (-10,-11);
\end{tikzpicture}\ \right)+3}
(-A)^{-3}
\left\langle\ \begin{tikzpicture}[scale=0.025,baseline=-4]
	\path[TEXTARC] (-10,7) -- (-10,-11);
\end{tikzpicture}\ \right\rangle =
(-A)^{-3
w\left(\ \begin{tikzpicture}[scale=0.025,baseline=-4]
	\path[TEXTARC] (-10,7) -- (-10,-11);
\end{tikzpicture}\ \right)}
\left\langle\ \begin{tikzpicture}[scale=0.025,baseline=-4]
	\path[TEXTARC] (-10,7) -- (-10,-11);
\end{tikzpicture}\ \right\rangle.\qedhere
\end{eqnarray*}
\end{proof}

Wielomian Jonesa jest naprawdę potężnym narzędziem.
Pozwala bowiem odróżnić dowolne dwa węzły pierwsze o co najwyżej dziewięciu skrzyżowaniach.

\begin{hipoteza}
Nie istnieje nietrywialny węzeł, którego wielomian Jonesa nie odróżnia od niewęzła.
\end{hipoteza}

\begin{twierdzenie}
Wielomianem węzła $(m, n)$-torusowego jest
\[
	\frac {t^{(m-1)(n-1):2}}{1-t^2} \cdot (1 - t^{m+1} - t^{n+1} + t^{m+n}).
\]
\end{twierdzenie}

\subsection{Relacja kłębiasta}
Dotychczas wyznaczyliśmy wielomian Jonesa jedynie dla trywialnych splotów.
Spowodowane jest to tym, że chociaż nawias Kauffmana jest przydatny przy dowodzeniu różnych własności, to zupełnie nie nadaje się do obliczeń.
Dużo lepszym narzędziem jest następujące twierdzenie.

\begin{twierdzenie}[relacja kłębiasta]
	Wielomian Jonesa spełnia równość $V(\NieWezel) = 1$ oraz relację
	\[
		t^{-1} V(L_+) - tV(L_-) + (t^{-1/2} - t^{1/2}) V(L_0) = 0,
	\]

	gdzie $L_+$, $L_-$, $L_0$ to zorientowane sploty, kóre różnią się jedynie na małym obszarze:
	$
		\begin{tikzpicture}[scale=0.03, baseline=-3]
		\path[TEXTARC,->] (-5,-5) -- (5,5);
		\path[TEXTARC] (5,-5) -- (1.5,-1.5);
		\path[TEXTARC,<-] (-5,5) -- (-1.5,1.5);
		\end{tikzpicture}
	$, 
	$
		\begin{tikzpicture}[scale=0.03, baseline=-3]
		\path[TEXTARC,->] (5,-5) -- (-5,5);
		\path[TEXTARC] (-5,-5) -- (-1.5,-1.5);
		\path[TEXTARC,<-] (5,5) -- (1.5,1.5);
		\end{tikzpicture}
	$, 
	$
		\begin{tikzpicture}[scale=0.03, baseline=-3]
		\path[TEXTARC,->] (-5,-5) .. controls (-1.5,-1.5) and (-1.5,1.5) .. (-5,5);
		\path[TEXTARC,->] (5,-5) .. controls (1.5,-1.5) and (1.5,1.5) .. (5,5);
		\end{tikzpicture}
	$.
\end{twierdzenie}

\begin{proof}
Wyraźmy wielomian Jonesa przez nawias Kauffmana i spin.
Chcemy pokazać, że
\[
A^{4}(-A)^{-3w(L_+)}\langle\PrawyKrzyz\rangle
-A^{-4}(-A)^{-3w(L_-)}\langle\LewyKrzyz\rangle
+(A^2-A^{-2})(-A)^{-3w(L_0)}\langle\PrawyGladki\rangle
=0.
\]
Ale $w(L_\pm)=w(L_0)\pm 1$, zatem to jest równoważne z $-A\langle\PrawyKrzyz\rangle +A^{-1}\langle\LewyKrzyz\rangle +(A^2-A^{-2})\langle\PrawyGladki\rangle =0$.
Z definicji nawiasu Kauffmana wnioskujemy, że
$\langle\PrawyKrzyz\rangle = A\langle\PrawyGladki\rangle+A^{-1}\langle\LewyGladki\rangle$ i $\langle\LewyKrzyz\rangle = A\langle\LewyGladki\rangle+A^{-1}\langle\PrawyGladki\rangle$.
Pierwsze równanie przemnóżmy przez $A$, drugie przez $A^{-1}$, a następnie dodajmy je do siebie.
Wtedy otrzymamy $A\langle\PrawyKrzyz\rangle-A^{-1}\langle\LewyKrzyz\rangle = A^2\langle\PrawyGladki\rangle - A^{-2}\langle\PrawyGladki\rangle$.
\end{proof}

\begin{przyklad}
	$V(
		\begin{tikzpicture}
		[scale=0.02, baseline=-3]
		\clip (-15,-9) rectangle (15,9);
		\path[TEXTARC,-<-] (1.5,-2.75) arc (-20:300:8);
		\path[TEXTARC,-<-] (-1.5,2.75) arc (160:480:8);
		\end{tikzpicture}) = -t^{5/2} - t ^{1/2}
	$ (splot Hopfa),
	$
	V(\begin{tikzpicture}[scale=0.02,baseline=-2]
	\clip (-19,-13) rectangle (19,17);
	\foreach \x in {270,30, 150}
		\path[TEXTARC,-<-] (15+\x:6) .. controls (130+\x:25) and (200+\x:25) .. (225+\x:10);
\end{tikzpicture}) = -t^4+t^3 + t
$ (trójlistnik).
\end{przyklad}

\subsection{Odwrotności, lustra i sumy}

\begin{twierdzenie}
Niech $L$ będzie zorientowanym splotem.
$V(rL)=V(L)$, $V(mL)(t)=V(L)(t^{-1})$.
\end{twierdzenie}

\begin{wniosek}
Wielomian Jonesa nie zależy od orientacji węzła (ale nie splotu!).
\end{wniosek}

\begin{proof}
Każdy węzeł ma tylko dwie orientacje, splot może mieć ich $2^n$, gdzie $n$ to liczba składowych.
\end{proof}

\begin{wniosek}
Trójlistnik nie jest równoważny ze swoim lustrem.
\end{wniosek}

\begin{proof}
W zależności od orientacji wielomianem trójlistnika jest $...$ lub $...$.
\end{proof}

\begin{twierdzenie}
Niech $L, M$ będą zorientowanymi splotami, zaś $J, K$: zorientowanymi węzłami.
\begin{enumerate}
\item $V(L \sqcup M) = (-t^{1/2} - t^{-1/2}) V(L) V(M)$,
\item $V(J \# K) = V(J) V(K)$.
\end{enumerate}
\end{twierdzenie}

\begin{proof}
Wybierzmy diagramy $D, E$ dla (odpowiednio) $L, M$.
Po podstawieniu $t^{1/2}=A^{-2}$ widzimy, że chcemy pokazać $(-A)^{-3w(D\sqcup E)}\langle D\sqcup E\rangle =(-A^2-A^{-2})(-A)^{-3(w(D)+w(E))}\langle D\rangle  \langle E\rangle$.

Oczywiście $w(D\sqcup E)=w(D)+w(E)$, więc wystarczy udowodnić, że 
\[
	\langle D\sqcup E\rangle = (-A^2-A^{-2})\langle D\rangle\langle E\rangle.
\]

Oznaczmy przez $f_1(D)$, $f_2(D)$ lewą i prawą stronę ostatniego równania.
Są to wielomiany Laurenta, które zależą tylko od $D$.
Aksjomaty Kauffmana pozwalają na pokazanie, że obie funkcje mają następujące własności:
$f_i(\NieWezel)=(-A^2-A^{-2})\langle E\rangle$, $f_i(D\sqcup\NieWezel)=(-A^2-A^{-2})f_i(D)$, $f_i(\PrawyKrzyz)=Af_i(\PrawyGladki) + A^{-1}f_i(\LewyGladki)$.
To pozwala na wyznaczenie ich wartości dla dowolnego $D$, zatem $f_1 \equiv f_2$, co kończy dowód.
\end{proof}

\begin{proof}
Narysujmy $J, K$ jako
$\begin{tikzpicture}[scale=0.02, baseline=3]
	\path[TEXTARC] (-40,0) rectangle (-20,20);
	\path[TEXTARC,-<-] (-20,17) .. controls (-5,17) and (-5,3) .. (-20,3);
	\node[darkblue] at (-30,10) {J};
	\path[TEXTARC] (35,0) rectangle (15,20);
	\path[TEXTARC,-<-] (15,17) .. controls (0,17) and (0,3) .. (15,3);
	\node[darkblue] at (25,10) {K};
\end{tikzpicture}$.
Rozpatrzmy sploty 
$\begin{tikzpicture}[scale=0.02, baseline=3]
	\path[TEXTARC] (-40,0) rectangle (-20,20);
	\node[darkblue] at (-30,10) {J};
	\path[TEXTARC] (30,0) rectangle (10,20);
	\path[TEXTARC] (10,3) -- (-3,9);
	\path[TEXTARC,->-] (-7,11) -- (-20,17);
	\path[TEXTARC] (-20,3) -- (-5,10);
	\path[TEXTARC,->-](-5,10) -- (10,17);
	\node[darkblue] at (20,10) {K};
\end{tikzpicture}
$, 
$\begin{tikzpicture}[scale=0.02, baseline=3]
	\path[TEXTARC] (-40,0) rectangle (-20,20);
	\node[darkblue] at (-30,10) {J};
	\path[TEXTARC] (30,0) rectangle (10,20);
	\path[TEXTARC] (-20,3) -- (-7,9);
	\path[TEXTARC,->-] (-3,11) -- (10,17);
	\path[TEXTARC] (10,3) -- (-5,10);
	\path[TEXTARC,->-] (-5,10) -- (-20,17);
	\node[darkblue] at (20,10) {K};
\end{tikzpicture}
$, 
$\begin{tikzpicture}[scale=0.02, baseline=3]
	\path[TEXTARC] (-40,0) rectangle (-20,20);
	\path[TEXTARC,-<-] (-20,17) .. controls (-5,17) and (-5,3) .. (-20,3);
	\node[darkblue] at (-30,10) {J};
	\path[TEXTARC] (35,0) rectangle (15,20);
	\path[TEXTARC,-<-] (15,17) .. controls (0,17) and (0,3) .. (15,3);
	\node[darkblue] at (25,10) {K};
\end{tikzpicture}$.
Relacja kłębiasta może zostać użyta do pokazania, że 
\[
t^{-1}V(J\#K)-tV(J\#K)+(t^{-1/2}-t^{1/2})V(J\sqcup K)=0.
\]
Ale $V(J\sqcup K)=(-t^{1/2}-t^{-1/2})V(J)V(K)$, co upraszcza się do $V(J\#K)=V(J)V(K)$ i kończy dowód.
\end{proof}

\subsection{Rozpiętość i wielomian Jonesa}

\begin{twierdzenie}
Niech $L$ posiada zredukowany, spójny, alternujący diagram o $n$ skrzyżowaniach.
Wtedy każdy diagram ma co najmniej $n$ skrzyżowaniach.
\end{twierdzenie}

To bardzo ważny rezultat, którego prawdziwość przypuszczał już P. G. Tait w XIX wieku.
Nikt nie był w stanie podać dowodu przed pojawieniem się wielomianu Jonesa w latach osiemdziesiątych.
Wyjaśnimy teraz użyte tu przymiotniki.

\begin{definicja}
Diagram jest alternujący, gdy podczas poruszania się wzdłuż splotu mijamy skrzyżowania na zmianę z góry oraz z dołu.
Diagram jest \emph{zredukowany}, gdy nie zawiera usuwalnych skrzyżowań.
Diagram jest \emph{spójny}, gdy nie można go podzielić na dwie niepuste części, które nie spotykają się na żadnym skrzyżowaniu.
\[\begin{tikzpicture}[scale=0.1]
	\path[ARC] (-5,-5) rectangle (5,5);
	\path[ARC] (-5,-3) -- (-12,0);
	\path[ARC] (-12,0)  -- (-19,3);
	\path[ARC] (-5,3) -- (-10,1);
	\path[ARC] (-14,-1) -- (-19,-3);
	\path[ARC,dotted] (-19,3) -- (-23.5,5);
	\path[ARC,dotted] (-19,-3) -- (-23.5,-5);
\end{tikzpicture}
\]
\end{definicja}

Przykładowo diagram $\begin{tikzpicture}
	[scale=0.02, baseline=-3]
	\path[TEXTARC] (0,0) circle (8);
	\path[TEXTARC] (20,0) circle (8);
\end{tikzpicture}$ nie jest spójny, ale $\begin{tikzpicture}
	[scale=0.02, baseline=-3]
	\path (1.5,-2.75) arc (-20:300:8) node (here) {};
	\path[TEXTARC] (here) arc (300:60:8);
	\path[TEXTARC] (-1.5,2.75) arc (160:520:8);
	\path[TEXTARC] (1.5,-2.75) arc (-20:20:8);
\end{tikzpicture}$ już tak.

W dowodzie przywołanego wyżej twierdzenia użyjemy rozpiętości wielomianu Jonesa.

\begin{definicja}
Niech $f$ będzie wielomianem Laurenta zmiennej $X$. Wtedy $M(f)$ [$m(f)$] to najwyższa [najniższa] potęga pojawiająca się w $f$. 
\emph{Rozpiętość} to $\operatorname{span} f = M(f) - m(f)$.
\end{definicja}

Zajmiemy się teraz nawiasem Kauffmana.
Znajdziemy wzór, który pozwala na wyznaczenie nawiasu dowolnego splotu o $n$ skrzyżowaniach (na diagramie) przez dodanie $2^n$ wyrazów.
Wzór ten okaże się użyteczny przy dowodzeniu późniejszych twierdzeń.




\begin{definicja}
Niech $D$ będzie diagramem splotu.
\begin{enumerate}
\item \emph{Stan} $D$ to funkcja $s$ ze zbioru skrzyżowań $D$ w $\{-1, +1\}$.
\item Dla ustalonego stanu $s$ dla $D$ przez $sD$ rozumiemy diagram powstały przez wygładzenie wszystkich skrzyżowań zgodnie z ich nowym znakiem ($\pm 1$), wtedy $|s|$ to suma wartości $s$.
\item Diagram dla $sD$ jest sumą zamkniętych krzywych, ich liczbę oznaczamy przez $|sD|$.
\end{enumerate}
\end{definicja}

\begin{twierdzenie}[o sumowaniu stanów]
Niech $D$ będzie diagramem splotu.
Wtedy
\[\langle D\rangle = \sum_s (-A^2-A^{-2})^{|sD|-1} A^{|s|},\]
gdzie sumujemy po wszystkich stanach $s$ dla $D$.
\end{twierdzenie}

\newpage


% dodać odwołanie
\begin{proof}
Założenia mówią nam, że $\operatorname{span} (V(L)) = n$.
Gdyby istniał diagram o mniejszej liczbie skrzyżowań, mielibyśmy $\operatorname{span} (V(L)) < n$, co prowadzi do sprzeczności.
\end{proof}


\newpage

\raggedright

\section*{Starsze materiały}

\textbf{Węzeł} to obraz różnowartościowej funkcji $f \colon S^1 \to \R^3$, która jest gładka i ma wszędzie niezerową pochodną. Przykłady: niewęzeł, trójlistnik, ósemka.
\textbf{Splot} to suma rozłączna skończenie wielu węzłów.
Węzły $K$ i $K'$ są równoważne, jeżeli istnieje rodzina węzłów $K_x$ dla $x \in [0,1]$, że $K_0 = K$ i $K_1 = K'$ oraz gładka funkcja $[0,1] \times \R \to \R^3$, że $F(x,$--$)$ reprezentuje węzeł $K_x$.

\textbf{Twierdzenie Reidemeistera}: każdy węzeł (splot) ma diagram, czyli ,,cień'' bez potrójnych skrzyżowań, stycznych i dziobów razem z informacją o skrzyżowaniach.
Dwa węzły są równoważne $\iff$ jeden da się otrzymać z drugiego przez ciąg ruchów Reidemeistera i gładkich deformacji.

\emph{Linking number}: suma znaków skrzyżowań, rozróżnia splot Hopfa od Whiteheada.
\emph{Crossing number}: najmniejsza możliwa liczba skrzyżowań na diagramie.

Nowe węzły: lustro, odwrócenie, suma rozłączna i suma spójna.

\textbf{Kolorowanie}: każdemu łukowi przypisujemy pewną liczbę całkowitą tak, by suma (znakowana) tych liczb na każdym skrzyżowaniu była równa zero modulo $n$.
Jest niezmiennikiem, odróżnia ósemkę od trójlistnika.
Macierz kolorowania i jej wyznacznik; macierz Goeritza.

\textbf{Grupa kolorująca}: abelowa grupa, której generatory to łuki diagramu, zaś relacje odpowiadają równaniom na skrzyżowaniach (do tego dokładamy $a = 0$).
Grupa jest skończona $\iff$ wyznacznik jest niezerowy.

\textbf{Wielomian Alexandera}: wyznacznik pewnej macierzy ze skreśloną kolumną i wierszem.
\textbf{Wielomian Jonesa} $V$: należy do $\mathbb Z[t^{1/2}, t^{-1/2}]$; wielomian niewęzła to $1$, spełniona jest \emph{skein relation}, to znaczy: $t^{-1}V(L_+)-tV(L_-)+(t^{-1/2}-t^{1/2})V(L_0)=0$.

Są jeszcze powierzchnie Seiferta z topologii algebraicznej.
Adams: notacja Dowkera, Conwaya. Węzły torusowe (5.1), węzły satelitarne i hiperboliczne.
Warkocze (5.4), wielomiany HOMFLY (6.3).
\emph{Bachelor's unknotting}.

Rzeczy raczej zbyt skomplikowane: niezmiennik Arfa.
\end{document}		